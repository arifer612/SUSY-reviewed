%%%%%%%%%%%%%%%%%%%% author.tex %%%%%%%%%%%%%%%%%%%%%%%%%%%%%%%%%%%
%
% sample root file for your "contribution" to a contributed volume
%
% Use this file as a template for your own input.
%
%%%%%%%%%%%%%%%% Springer %%%%%%%%%%%%%%%%%%%%%%%%%%%%%%%%%%


% RECOMMENDED %%%%%%%%%%%%%%%%%%%%%%%%%%%%%%%%%%%%%%%%%%%%%%%%%%%
\documentclass[%
graybox,
vecphys,
]{svmult}

% choose options for [] as required from the list
% in the Reference Guide

\usepackage{type1cm}        % activate if the above 3 fonts are
                            % not available on your system
%
\usepackage{makeidx}         % allows index generation
\usepackage{graphicx}        % standard LaTeX graphics tool
                             % when including figure files
\usepackage{multicol}        % used for the two-column index
\usepackage[bottom]{footmisc}% places footnotes at page bottom


\usepackage{newtxtext}       % 
\usepackage{newtxmath}       % selects Times Roman as basic font

% see the list of further useful packages
% in the Reference Guide

\makeindex             % used for the subject index
                       % please use the style svind.ist with
                       % your makeindex program

%%%%%%%%%%%%%%%%%%%%%%%%%%%%%%%%%%%%%%%%%%%%%%%%%%%%%%%%%%%%%%%%%%%%%%%%%%%%%%%%%%%%%%%%%
% USER DEFINITIONS

%USER PACKAGES
\usepackage{slashed}        % Dirac slash notation
\usepackage{braket}         % Braket notation
\usepackage{bbold}

\newcommand{\nonext}{\nonumber \\}  % Used for eqnarrays
\newcommand{\td}{\text{d}}          % For differentials
\newcommand{\e}[1]{\text{e}^{#1}}   % Exponential
\newcommand{\lc}{\left(}            % Left bracket formatting
\newcommand{\rc}{\right)}           % Right bracket formatting
\newcommand{\iden}{\mathbb{1}}      % Identity
\newcommand{\col}[2]{\begin{pmatrix}{#1}\\{#2}\end{pmatrix}}
\newcommand{\diff}[2]{\frac{\partial {#1}}{\partial {#2}}}


%%%%%%%%%%%%%%%%%%%%%%%%%%%%%%%%%%%%%%%%%%%%%%%%%%%%%%%%%%%%%%%%%%%%%%%%%%%%%%%%%%%%%%%%%

\begin{document}

\title*{Review of Supersymmetry}
% Use \titlerunning{Short Title} for an abbreviated version of
% your contribution title if the original one is too long
\author{Arif Er}
% Use \authorrunning{Short Title} for an abbreviated version of
% your contribution title if the original one is too long
\institute{Arif Er \at NUS, \email{e0032348@u.nus.edu}}
%
% Use the package "url.sty" to avoid
% problems with special characters
% used in your e-mail or web address
%
\maketitle

\abstract{A review and revision of my study of Supersymmetry I studied in the summer of 2020. This review will cover the topics covered in \textit{Supersymmetry DeMYSTiFied} by Labelle \cite{labelle2010supersymmetry}}

\setcounter{chapter}{1}
\chapter{Introduction to Weyl spinors}
\label{ch:2}

We begin the revision by recapping on the physics of the Weyl spinors and how they are relevant in our study of the Quantum Field Theories. We will then attempt to formulate possible Lorentz invariants from the Dirac spinors so that they can be used in the Lagrangian formalism. Lastly, we will look at the Van der Waarden notation, a more compact and useful notation for Weyl spinors especially in the context of Supersymmetry.

\section{The Dirac equation}
\label{ch:1:dirac equation}
Our starting point is the Dirac equation. It relates shows how one can obtain the eigenvalue of the momentum operator of a quantum particle.
\begin{equation}
    \gamma^\mu P_\mu \psi = m \psi \quad , P_\mu \equiv i \partial_\mu
    \label{eqn:2:dirac equation}
\end{equation}

Using the Dirac slash, it is identically
\begin{equation}
    \slashed P \psi = m \psi
    \label{eqn:2:slashed dirac equation}
\end{equation}

The Lagrangian for a Dirac particle is thus
\begin{equation}
    \mathcal{L}_{Dirac} = \Bar{\psi} (\gamma^\mu P_\mu - m) \psi
    \label{eqn:2:dirac lagrangian}
\end{equation}

The $\gamma^\mu$ used above are the Dirac matrices, $4\times4$ matrices that are built off the Pauli matrices.
\begin{equation}
    \gamma^0 =
    \begin{pmatrix}
        0 & \mathbb{1}\\
        \mathbb{1} & 0
    \end{pmatrix}
    \quad , \quad
    \gamma^i =
    \begin{pmatrix}
        0 & - \sigma^i\\
        \sigma^i & 0
    \end{pmatrix}
    \label{eqn:2:dirac matrices}
\end{equation}

Using the mostly negative signature metric (i.e. $\eta_\mu\nu = (+, -, -, -)$), the Dirac matrices are:
\begin{equation}
    \gamma^\mu = (\gamma^0, \gamma^i) \quad , \quad\gamma_\mu = (\gamma^0, -\gamma^i)
    \label{eqn:2:dirac matrices lowered}
\end{equation}

We were also introduced another new Dirac matrix, for the fact that it simplifies a large deal of work in the later part of our journey.
\begin{equation}
    \gamma_5 =
    \begin{pmatrix}
        \mathbb{1} & 0\\
        0 & \mathbb{1}
        \label{eqn:2:dirac matrix 5}
    \end{pmatrix}
\end{equation}

From the properties of Pauli matrices, we see some interesting results that would turn out to be central to the formulation of the framework.
\begin{eqnarray}
    \sigma^2 (\sigma^i)^T &=& - (\sigma^i) \sigma^2 \nonext
    \sigma^2 (\sigma^i)^* &=& - (\sigma^i) \sigma^2
    \label{eqn:2:pauli matrix property 1}
\end{eqnarray}
\begin{eqnarray}
    \sigma^2 (\sigma^i) \sigma^2 = - (\sigma^i)^T = - (\sigma^i)^*\nonext
    \therefore \sigma^2 (\sigma^i)^T \sigma^2 = - (\sigma^i)
    \label{eqn:2:sigma T and *}
\end{eqnarray}

We thus have the following representation of a vector weighted matrix
\begin{eqnarray}
    \Vec{A} \cdot \Vec{\sigma} \sigma^j &=& A^i \sigma^i \sigma^j \nonext
    &=& A^i (\sigma^j \sigma^i - [\sigma^i, \sigma^j]) \nonext
    &=& A^i \sigma^j \sigma^i - 2 i \varepsilon^{ijk}A^i \sigma^k \nonext
    &=& \sigma^j \Vec{A} \cdot \Vec{\sigma} - 2 i (\Vec{A} \times \Vec{\sigma})
    \label{eqn:2:vector weighted matrix}
\end{eqnarray}

\section{Dirac spinors}
These are reducible 4 component spinors. Their lowest representation is a 2 component spinor, a `left-chiral' and a `right-chiral' spinor. 
\begin{equation}
    \psi = \begin{pmatrix} \eta \\ \chi \end{pmatrix}
    \label{eqn:2:dirac spinor}
\end{equation}

Extracting the irreducible spinors is simple with a projection operator. Clearly, if
\begin{equation*}
    P_R \psi = \begin{pmatrix} \eta \\ 0 \end{pmatrix} 
    \quad , \quad
    P_L \psi = \begin{pmatrix} 0 \\ \chi \end{pmatrix}
\end{equation*}
The projection operators have to be:
\begin{equation*}
    P_R = \begin{pmatrix} \mathbb{1} & 0 \\ 0 & 0 \end{pmatrix}
    \quad , \quad
    P_L = \begin{pmatrix} 0 & 0 \\ 0 & \mathbb{1} \end{pmatrix}
\end{equation*}
which relates back to the $\gamma_5$ matrix in Eqn \ref{eqn:2:dirac matrix 5} as
\begin{equation}
    \gamma_5 = P_R - P_L
\end{equation}

We will use this to expand the Dirac equation into its irreducibles.
\begin{eqnarray}
    \gamma^\mu P_\mu \psi = m \psi
    &\implies&
    \begin{cases}
        (E \mathbb{1} + \Vec{\sigma} \cdot \Vec{p}) \chi = m \eta\\
        (E \mathbb{1} - \Vec{\sigma} \cdot \Vec{p}) \eta = m \chi
    \end{cases}
    \label{eqn:2:dirac equation in weyl spinors}
    \\
    &=&
    \begin{cases}
        \Bar{\sigma}^\mu P_\mu \chi = m \eta \\
        \sigma^\mu P_\mu \eta = m \chi
    \end{cases}
    \label{eqn:2:dirac equation in weyl spinors compact}
\end{eqnarray}
where we are introduced to $\sigma^\mu = (\mathbb{1}, \sigma^i)$ and $\Bar{\sigma}^\mu = (\mathbb{1}, - \sigma^i)$ to get to Equation \ref{eqn:2:dirac equation in weyl spinors compact} from Equation \ref{eqn:2:dirac equation in weyl spinors}

The Dirac Lagrangian in its irreducible forms is thus
\begin{equation}
    \mathcal{L}_{Dirac} = \chi^\dagger i \Bar{\sigma}^\mu \partial_\mu \chi + \eta^\dagger i \sigma^\mu \partial_\mu \eta - m(\chi^\dagger \eta + \eta^\dagger \chi)
    \label{eqn:2:dirac lagrangian in weyl spinors}
\end{equation}

This reveals the coupled nature of the Dirac spinor irreducibles, or at least if they are massive. The special case of the massless Dirac spinor are known Weyl spinors. Being massless, we get the Weyl equations that tell us the eigenvalues of the spinors with respect to the $\Vec{\sigma}\cdot\Vec{p}$ operator:
\begin{equation}
    \begin{cases}
        E \eta = \Vec{\sigma} \cdot \Vec{p} \eta\\
        E \chi = - \Vec{\sigma} \cdot \Vec{p} \chi
    \end{cases}
    \implies
    \begin{cases}
        \frac{\Vec{\sigma}\cdot\Vec{p}}{\vert p \vert} \eta = \eta\\
        \frac{\Vec{\sigma}\cdot\Vec{p}}{\vert p \vert} \chi = \chi\\
    \end{cases}
    \label{eqn:2:weyl equations}
\end{equation}

And very conveniently, making use of the fact that $\Vec{S}\cdot\hat{p}$ is the helicity operator, we see why $\eta$ and $\chi$ are called chiral spinors!
\begin{equation}
    \Vec{S} \cdot \hat{p} \eta = \frac{\hbar}{2} \eta
    \quad , \quad 
    \Vec{S} \cdot \hat{p} \chi = - \frac{\hbar}{2} \chi
    \label{eqn:2:helical spinors}
\end{equation}

\section{Lorentz invariances}
\label{ch:2:lorentz invariances}
We want to build Lorentz invariants made of Weyl spinors, so that we might add them as interactions when building Lagrangians of any theory that might come along. For this, we need to know how they transform. From the fact that the Dirac Lagrangian is necessarily Lorentz invariant, the terms of Equation \ref{eqn:2:dirac lagrangian in weyl spinors} also has to be. Let us look into each term carefully.

\subsection{Pure terms}
\label{ch:2:lorentz invariances:pure terms}
The `pure' terms are $\chi^\dagger\eta$ and $\eta^\dagger\chi$. We know how each spinor transforms under the Lorentz group:
\begin{eqnarray}
    \eta &\rightarrow& \lc\mathbb{1} + \frac{1}{2} i \Vec{\varepsilon}\cdot\Vec{\sigma} - \frac{1}{2} \Vec{\beta}\cdot\Vec{\sigma}\rc \eta
    \label{eqn:2:right chiral transformation} \\
    \chi &\rightarrow& \lc\mathbb{1} + \frac{1}{2} i \Vec{\varepsilon}\cdot\Vec{\sigma} + \frac{1}{2} \Vec{\beta}\cdot\Vec{\sigma}\rc \chi 
    \label{eqn:2:left chiral transformation}
\end{eqnarray}
The difference between the transformation of the left and right chiral spinors is very subtle. But this difference allows us to understand why the `pure' terms are invariant. Moving on to the remainder of the terms,

\subsection{Mixed terms}
\label{ch:2:lorentz invariances:mixed terms}
These are the terms with $\sigma^\mu$ in them: $\chi^\dagger i \Bar{\sigma}^\mu \partial_\mu \chi$ and $\eta^\dagger i \sigma^\mu \partial_\mu \eta$. We know from earlier that $\chi^\dagger\eta$ forms an invariant. Therefore the term that couples with $\chi^\dagger$ has to transform like a right chiral spinor. This means that $i\Bar{\sigma^\mu}\partial_\mu\chi$ transforms like a right chiral spinor. This is trivial to prove. One way is to recall the coupled irreducibles in Equation \ref{eqn:2:dirac equation in weyl spinors compact}. $m$ is an invariant, so the remaining terms have to transform in the same manner as each other. The other way is to explicitly find the transformation rule of $i \Bar{\sigma}^\mu \partial_\mu \chi$ to find out that it does indeed transform as a right chiral spinor. It is the same as $i \sigma^\mu \partial_\mu \eta$, which transforms neatly as a left chiral spinor.

Moreover, if we were to do an integration by parts (IbP) on these mixed terms, we will find that:
\begin{equation}
    - \partial_\mu(\chi^\dagger i \Bar{\sigma}^\mu) \chi = (i \Bar{\sigma}^\mu \partial_\mu \chi)^\dagger \chi
    \label{eqn:2:invariant ibp}
\end{equation}
is also in invariant! Clearly for Equation \ref{eqn:2:invariant ibp} to make sense with the results in the Section \ref{ch:2:lorentz invariances:pure terms} $i \Bar{\sigma}^\mu \partial_\mu \chi$ has to transform as a right chiral, which again agrees with what we did earlier.

\subsection{Using only left chirals}
\label{ch:2:lorentz invariances:using sigma2}
Using the properties of $\sigma^2$, we can create Lorentz invariants using only left chirals, without the need of any vector matrices. The transformation of ${\chi^\dagger}^T$ in the contraction $\chi^\dagger\eta$ is
\begin{eqnarray}
    {\chi^\dagger}^T &\rightarrow& \lc \iden + \frac{1}{2} i \Vec{\varepsilon} \cdot \Vec{\sigma} + \frac{1}{2} \Vec{\beta} \cdot \Vec{\sigma} \rc^* {\chi^\dagger}^T \nonext
    &=& \lc \iden - \frac{1}{2} i \Vec{\varepsilon} \cdot \Vec{\sigma} + \frac{1}{2} \Vec{\beta} \cdot \Vec{\sigma} \rc {\chi^\dagger}^T
    \label{eqn:2:left chiral dagger transformation}
\end{eqnarray}

To pair this with the left chiral $\chi$, we need to get it to transform as a right chiral, for which is Equation \ref{eqn:2:left chiral dagger transformation} but with the opposite sign. Multiplying throughout by $i \sigma^2$ does exactly that, as we did in Equation \ref{eqn:2:pauli matrix property 1}. Therefore, $i \sigma^2 {\chi^\dagger}^T$ transforms like a right chiral!

The invariant we will get out of this combination is done by taking the hermitian conjugate of it and pairing it with $\chi$.
\begin{equation}
    \lc i \sigma^2 {\chi^\dagger}^T \rc^\dagger = \chi^T \lc -i \sigma^2 \rc \chi
    \label{eqn:2:left chirals invariant}
\end{equation}

\subsection{Method}
\label{ch:2:lorentz invariances:method}
From what we have seen so far, Lorentz invariants made of Weyl spinors all require a combination of one left and right chiral spinor, with either of them being a hermitian conjugate. We can extend this further by making use of the fact that $\chi^\dagger \Bar{\sigma}^\mu \chi$ is a (1,0) tensor, since $P_\mu$ necessarily transforms as a vector under the Lorentz group.

\begin{question}{Creating a rank 2 covariant tensor from left chirals, without derivatives}
    This is a rather trivial exercise if we work out how the rank 1 covariant tensor $\chi^\dagger \Bar{\sigma}^\mu \chi$ transforms. Let us define the transformation of $\chi$ as $\chi \rightarrow A^{-1} \chi $, where A is naturally a transformation matrix.
    
    \begin{eqnarray*}
        \chi^\dagger \Bar{\sigma}^\mu \chi \rightarrow {\chi'}^\dagger \Bar{\sigma'}^\mu \chi' &=& \chi^\dagger {A^{-1}}^\dagger \Bar{\sigma}^\mu A^{-1} \chi
        \nonext
        &=& \chi^\dagger \Lambda^\mu_\nu \Bar{\sigma}^\nu\chi
    \end{eqnarray*}
    
    From this, we know that the transformation of $\Bar{\sigma}^\mu$ is
    \begin{equation*}
        \Bar{\sigma}^\mu \rightarrow A^\dagger \Lambda^\mu_\nu \Bar{\sigma}^\nu A
    \end{equation*}
    Likewise for $\sigma^\mu$, since $\eta^\dagger \sigma^\mu \eta$ is a rank 1 covariant tensor and $\eta^\dagger\chi$ is invariant (i.e. $\eta \rightarrow A^\dagger \eta$
    \begin{equation*}
        \sigma^\mu \rightarrow A^{-1} \Lambda^\mu_\nu \sigma^\nu {A^{-1}}^\dagger
    \end{equation*}
    
    Putting them together, we have
    \begin{equation*}
        \sigma^\mu \Bar{\sigma}^\nu \rightarrow A^{-1}\Lambda^\mu_\alpha\Lambda^\nu_\beta \sigma^\alpha \sigma^\beta A
    \end{equation*}
    
    Fitting this with a something that transforms as $? \rightarrow A^\dagger$ (i.e. a right chiral spinor) on the left and $? \rightarrow A^{-1}$ (i.e. a left chiral spinor) on the right, we will get a rank 2 covariant tensor! Since we are looking to populate these positions with only left chirals, the options are obvious.
    \begin{equation*}
        \chi^T (-i \sigma^2) \sigma^\mu \Bar{\sigma}^\nu \chi \rightarrow \Lambda^\mu_\alpha \Lambda^\nu_\beta \chi^T \lc -i \sigma^2\rc \sigma^\alpha \Bar{\sigma}^\beta \chi
    \end{equation*}
\end{question}

\section{Van der Waerden notation}
\label{ch:2:van der waerden notation}
Instead of the cumbersome $\pm i \sigma^2$ in between the chiral spinors, the Van der Waerden notation defines a new kind of dot product between chiral spinors. We saw how $\chi^T (-i \sigma^2) \chi$ and $\eta^\dagger (i \sigma^2) {\eta^\dagger}^T$ are invariants, so we shall define 2 dot products as such:
\begin{eqnarray}
    \chi \cdot \chi &\equiv& \chi^T (-i \sigma^2) \chi \nonext
    \Bar{\chi} \cdot \Bar{\chi} &\equiv& \chi^\dagger (i \sigma^2) {\chi^\dagger}^T \nonumber
\end{eqnarray}

Expanding the components,
\begin{eqnarray}
    \chi \cdot\chi = \chi_2 \chi_1 - \chi_1\chi_2 \nonext
    \Bar{\chi} \cdot \Bar{\chi} = \chi^\dagger_1 \chi^\dagger_2 - \chi^\dagger_2 \chi^\dagger_1
    \label{eqn:2:van der waerden expanded}
\end{eqnarray}

The good thing about this notation is that it shows that the dot product is now hermitian! So we can employ this directly in the Lagrangian fully knowing that we need not worry about any real-ness violations.
\chapter{A new notation}
\label{ch:3}
In this chapter, we will introduce a new notation for Weyl spinors that will be very useful in reading off the Lagrangian for interpretation and when working in the superfield approach.

\section{Indices}
\label{ch:3:indices}
We will begin off by defining the index notation. As we will mostly be working in the left-chiral basis, it will be the basis of which our notation will be built off from. We will begin from the invariant $\eta^\dagger\chi$. We will define as a contraction of
\begin{equation}
    \eta^\dagger \chi \equiv \eta^\dagger \chi_a
    \label{eqn:3:first definition}
\end{equation}

The hermitian conjugate of the right chiral spinor is defined as having an upper un-dotted index. We will define the right chiral spinor as having an upper dotted index.
\begin{equation}
    \eta \equiv \Bar{\eta}^{\Dot{a}} \implies \eta^a \equiv \lc \Bar{\eta}^{\Dot{a}}\rc ^\dagger
\end{equation}

With this, we can generalise the rest and see that the other fundamental invariant is
\begin{equation}
    \chi^\dagger\eta \equiv \Bar{\chi}_{\Dot{a}}\Bar{\eta}^{\Dot{a}}
\end{equation}

We will adopt a notation convention, that contractions between un-dotted indices are carried from top down, and for dotted indices from bottom up, as we see in the two definitions above.

\section{Raising and lowering indices}
We will make use of the fact that $(-i\sigma^2)_{ba}(i\sigma^2)^{ab} = \mathbb{1}$ allows us to define a metric to raise and lower the indices. To raise the indices,
\begin{equation}
    \Bar{\chi}^{\Dot{a}} \equiv (i\sigma^2)^{\Dot{a}b}\chi_b^\dagger = (i\sigma^2)^{\dot{a}\dot{b}}\Bar{\chi}_{\dot{b}}
    \label{eqn:3:raising operation}
\end{equation}
With this, we can see that explicitly,
\begin{equation}
    \Bar{\chi}^{\dot{1}} = \Bar{\chi}_{\dot{2}} = \chi_2^\dagger \quad , \quad \Bar{\chi}^{\dot{2}} = - \Bar{\chi}_{\dot{1}} = -\chi_1^\dagger
    \label{eqn:3:explicit raising operation}
\end{equation}

Likewise, lowering the indices is just as similar:
\begin{equation}
    \chi_b = (-i\sigma^2)_{ba}\chi^a
    \label{eqn:3:lowering operation}
\end{equation}

With this, the Van der Waerden dot product is
\begin{equation}
    \eta \cdot \chi = \eta^1 \chi_1 + \eta^2 \chi_2 = \eta_2\chi_1 - \eta_1\chi_2
\end{equation}

\section{The epsilon metric}
\label{ch:3:the epsilon metric}
To clean up the $(i\sigma^2)$ that is plaguing our notation, let us define
\begin{eqnarray}
    \varepsilon^{ab} &\equiv& (i\sigma^2)^{ab} \nonext
    \varepsilon^{\dot{a}\dot{b}} &\equiv& (i\sigma^2)^{\dot{a}\dot{b}} \nonext
    \varepsilon_{ab} &\equiv& (-i\sigma^2)_{ab} \nonext
    \varepsilon_{\dot{a}\dot{b}} &\equiv& (-i\sigma)^2_{\dot{a}\dot{b}} 
\end{eqnarray}

As $i\sigma^2$ is completely anti-symmetric, so is $\varepsilon$. Moreover, the contraction of $\varepsilon$ with itself is naturally
\begin{equation}
    \varepsilon^{ab}\varepsilon_{bc} = -\varepsilon^{ba}\varepsilon_{bc} = - \varepsilon^{ab}\varepsilon_{cb} = \varepsilon^{ba}\varepsilon_{cb} = \delta^a_c
\end{equation}

\section{$\sigma^\mu$ and $\Bar{\sigma}^\mu$ indices}
\label{ch:3:sigma indices}
We know that $i\sigma^\mu \eta$ is a left chiral spinor, so $i \sigma^\mu$ has to lower a dotted index to an un-dotted one. Likewise, $i\Bar{\sigma}^\mu \chi$ is a right chiral spinor, so $i \Bar{\sigma}^\mu$ raises an un-dotted index to a dotted one. We have:
\begin{equation}
    \Bar{\sigma}^\mu \equiv \left(\Bar{\sigma}^\mu\right)^{\dot{a}b} \quad , \quad \sigma^\mu \equiv \left(\sigma^\mu\right)_{a\dot{b}}
\end{equation}

We can see obtain this through $\varepsilon$
\begin{eqnarray}
    (i \sigma^2) \sigma^\mu (i \sigma^2) &=& \varepsilon^{ca}(\sigma^\mu)_{a\dot{b}}\varepsilon^{\dot{b}\dot{d}} \nonext
    &=& (\Bar{\sigma}^\mu)^{c\dot{d}} \nonext
    &=& - (\bar{\sigma}^\mu)^{\dot{d}c} \nonext
    &=& - \left(\bar{\sigma}^\mu\right)^T
\end{eqnarray}

With this, we can that supersymmetric invariants with $\sigma^\mu$ or $\bar{\sigma}^\mu$ needs to have spinors with both dotted and un-dotted indices, of the same generation:
\begin{equation}
    \bar{\chi}_{\dot{a}} \left(\Bar{\sigma}^\mu\right)^{\dot{a}b} \lambda_b \quad , \quad \chi^a (\sigma^\mu)_{a\dot{b}} \bar{\lambda}^{\dot{b}}
\end{equation}
\chapter{Weyl, Majorana, and Dirac spinors}
\label{ch:4}

In this chapter, we look at close relations between Weyl, Majorana, and Dirac spinors and how we can jump between one to the other (and when we should not). The advantages of understanding this is that it helps paint a clearer picture behind the interpretation of these rather `abstract' representations of what particles are. 

\section{Particle-antiparticle}
\label{ch:4:particle-antiparticle}
We will begin this chapter by looking at the intricacies of particle-antiparticle existence. The source of their coexistence is by the natural imposition of charge-parity-time-reversal (CPT) invariance on the theory of particles. The particle-antiparticle pair would ensure that total charge, parity, and time-reversal is upheld. In the following text, the particle will be denoted by the subscript $p$ whereas the antiparticle will be denoted by the subscript ${\Bar{p}}$. Their representations as Dirac spinors (in terms of the irreducible Weyl spinors) are
\begin{equation*}
    \psi_p = \col{\eta_p}{\chi_p} \quad , \quad \psi_{\Bar{p}} = \col{\eta_{\Bar{p}}}{\chi_{\Bar{p}}}
\end{equation*}

It should be necessary to mention that it is not the case that $\eta_p = \eta_{\Bar{p}}$ and $\chi_p=\chi_{\Bar{p}}$.

The relation between the conjugate Dirac spinor and its barred transpose used in the Lagrangian is
\begin{equation}
    \psi_{\Bar{p}} = \psi_p^C = C {\Bar{\psi}_p}^T \quad , \quad 
    C = - i \gamma^2 \gamma^0 
    = \begin{pmatrix} i \sigma^2 & 0 \\0 & - i\sigma^2\end{pmatrix}
    \label{eqn:4:particle-antiparticle:conjugate relations}
    \nonumber
\end{equation}

With this, 
\begin{equation}
    \psi_p^C = \col{i \sigma^2 {\chi_p^\dagger}^T}{-i\sigma^2{\eta_p^\dagger}^T}
    \implies 
    \begin{cases}
        \eta_{\Bar{p}} = i \sigma^2 {\chi_p^\dagger}^T \\
        \chi_{\Bar{p}} = - i \sigma^2 {\eta_p^\dagger}^T
    \end{cases}
    \label{eqn:4:particle-antiparticle:particle-antiparticle relations}
\end{equation}

We now see how intricately related the Weyl spinors of the particle and antiparticle are. One very important thing to point out is that from Equation \ref{eqn:4:particle-antiparticle:particle-antiparticle relations}, we can see that our discussion in Sec \ref{ch:2:lorentz invariances:using sigma2} agrees that $i \sigma^2 {\chi_p^\dagger}^T$ behaves as a right chiral and $- i \sigma^2 {\eta_p^\dagger}^T$ as a left chiral! Using this knowledge, we can get rid of any explicit right chiral representations in the Dirac spinor and simply express it as
\begin{equation}
    \psi = \col{i \sigma^2 {\chi_{\Bar{p}}^\dagger}^T}{\chi_p}
    \label{eqn:4:particle-antiparticle:dirac spinor as left chirals}
    \nonumber
\end{equation}

As mentioned at the start, the particle-antiparticle relations only exists because of the CPT invariance imposed on the Lagrangian. The other necessary constraint is that the Lagrangian needs to be real. (i.e. $\mathcal{L^\dagger} = \mathcal{L}$) This constraint tells us that the Lagrangian should either have both the hermitian conjugates of any chiral spinors, or none at all. The interpretation of this in QFT is very physical. There has to either have both the annihilation and creation operator of a particle, or none at all. Number operators will thus be a conserved operation. 

The simplest contributor to the Lagrangian that has both left chiral spinors and ensures CPT and reality invariance is
\begin{equation*}
    \mathcal{L} = \chi^\dagger i \sigma^\mu \partial_\mu \chi
\end{equation*}
There is both a left chiral particle spinor creation and annihilation field operator in this Lagrangian.

However, using the relation in Equation \ref{eqn:4:particle-antiparticle:particle-antiparticle relations}, the same Lagrangian then becomes
\begin{equation*}
    \mathcal{L} = \eta_{\Bar{p}}^T (i \sigma^2) i \sigma^\mu \partial_\mu \chi
\end{equation*}
which is a left chiral particle spinor and right chiral antiparticle spinor creation field operator! A very thought-provoking interpretation of the particle-antiparticle relationship.

Insofar as we have used the term Weyl spinor, we have used it to identify particles that are both eigenstates of the helicity operator and the chiralty operator. However, there is a very subtle difference between the two that paints very different pictures of what a Weyl spinor really is. As eigenstates of the helicity operator, Weyl spinors are necessarily massless as shown in \ref{eqn:2:helical spinors}. However, as eigenstates of the chiralty operator, they are simply eigenstates with fixed transformation rules under the $SU(2) \times SU(2)$ Lorentz group as in Equations \ref{eqn:2:right chiral transformation} and \ref{eqn:2:left chiral transformation}. Thus, there is no constraint on them being massless. They can be as massive as they need be, as long as they are eigenstates that of the Lorentz group. However, for the sake of continuing the discussion regarding massive particles using the Weyl spinor representation, we shall adopt the convention of the latter, while duly keeping in mind that actual Weyl spinors are necessarily massless.

Returning to the CPT invariance, we now see that we have a scheme that relates $\eta_p$ to $\chi_{\Bar{p}}$ and its conjugates. Through the Lorentz transformation, it is also possible (\textbf{for massive particles}) for the chiralty of the particle to change, i.e from $\eta_p$ to $\chi_p$ and vice versa. These 4 particles are thus related to each other as a multiplet that must exist as a collective state. It is because of this fact that we are allowed to express the right chiral particle as the left chiral antiparticle with impunity. This is evident in how the mass term of the Lagrangian can be expressed in either of the following representations:
\begin{eqnarray}
    m \Bar{\psi}\psi
    &=& m (\chi^\dagger \eta + \eta^\dagger \chi) \nonext
    &=& m (\chi\cdot\chi + \Bar{\chi}\cdot\Bar{\chi})
\end{eqnarray}

\section{Majorana spinors}
\label{ch:4:majorana}
The Majorana is a special subset of (massive) Dirac spinors. Its antiparticle state is the same as its particle state, i.e. $\eta_p = \eta_{\Bar{p}}$ and $\chi_p = \chi_{\Bar{\chi}}$. Unlike the general Dirac spinor, we now have 2 degrees of freedom instead of 4. The Majorana spinor in left chiral representation is
\begin{equation}
    \psi_M = \col{i \sigma^2 {\chi_p^\dagger}^T}{\chi_p}
\end{equation}

As good as the Majorana and Weyl representations are, it is not possible to build actual theories using them only as parity is not conserved. In the Lagrangian formalism of strictly Weyl or Majorana spinors, the mass terms will only be mass terms of left chirals, with no way of satisfying the parity between left and right chirals.

Looking at the two from another angle, we see that
\begin{equation*}
    \begin{cases}
        \Bar{\psi}_M \psi_M = \chi\cdot\chi + \Bar{\chi}\cdot\Bar{\chi} \\
        \Bar{\psi}_M \gamma_5 \psi_M = - \chi\cdot\chi + \Bar{\chi}\cdot\Bar{\chi}
    \end{cases}
\end{equation*}
which with some simple manipulation and generalisation, simply gives us
\begin{equation*}
\begin{cases}
    \lambda\cdot\chi = \Bar{\Lambda}_M P_L\psi_M\\
    \Bar{\lambda}\cdot\Bar{\chi} = \Bar{\Lambda}_M P_R \psi_M
\end{cases}
\end{equation*}

Lastly, making use of the fact that $\gamma^\mu$ can be represented off-diagonally as
\begin{equation}
    \gamma^\mu = \begin{pmatrix} 0 & \Bar{\sigma}^\mu \\ \sigma^\mu & 0 \end{pmatrix}
\end{equation}
we have

\begin{equation*}
    \begin{cases}
        \Bar{\psi}_M \gamma^\mu \Lambda_M = \chi^\dagger \Bar{\sigma}^\mu \lambda - \lambda^\dagger \Bar{\sigma}^\mu \chi \\
        \Bar{\psi}_M \gamma_5\gamma^\mu \Lambda_M = \chi^\dagger \Bar{\sigma}^\mu \lambda + \lambda^\dagger \Bar{\sigma}^\mu \chi
    \end{cases}
\end{equation*}
\begin{equation*}
    \implies
    \begin{cases}
        \chi^\dagger \Bar{\sigma}^\mu \lambda = \Bar{\psi}_M P_R \gamma^\mu \Lambda_M\\
        \lambda^\dagger \Bar{\sigma}^\mu \chi = - \Bar{\psi}_M P_L \gamma^\mu \Lambda_M
    \end{cases}
\end{equation*}
a very neat relation between the Weyl and Majorana representations of the \textbf{massive} Majorana spinor.
\chapter{Building the Lagrangian}
\label{ch:5}

We will build attempt at building the most basic Lagrangians with the invariants and constraints from the previous chapters. Several constraints on the Lagrangian will have to be imposed to ground our discussion in renormalisable theories.

\section{Dimensionfull Lagrangians}
In building the Lagrangian, we shall keep to renormalisable theories where $D = 4$ is maximally the further we will go in dimensions. The dimensions are defined in terms of powers of energy and as should be, the natural units are 1 (thus being dimensionless). With this, scalar fields are of dimension 1, derivatives are of dimension 1, fermion fields are of dimension $3/2$. Simple dimensional analysis will give us these.

\section{The simplest Lagrangian}
Let us consider the simplest toy model we can make -- a single free massless fermionic pair and a single free massless bosonic pair. They do not interact with each other (this will be introduced in Chapter \ref{ch:6}). The Lagrangian is simply
\begin{equation}
    \mathcal{L} = \partial_\mu \phi \partial^\mu \phi^\dagger + \chi^\dagger i \Bar{\sigma}^\mu \partial_\mu \chi
\end{equation}

The transformation of these fields are
\begin{eqnarray}
    \phi &\rightarrow& \phi + \delta \phi \nonext
    \chi &\rightarrow& \chi + \xi \chi \nonumber
\end{eqnarray}

Let us bring in the postulate of supersymmetry -- that bosons will transform into fermions and vice versa.
\begin{equation}
    \delta \phi \propto \xi \chi \quad , \quad \xi \ll 1
    \label{eqn:5:boson transformation approx}
\end{equation}

For Equation \ref{eqn:5:boson transformation approx} to satisfy the dimensionality of both sides of the equation, we see that $\xi$ has to be a Grassmann spinor of dimension $-1/2$. To actually determine the proportionality of the relationship in Equation \ref{eqn:5:boson transformation approx}, we have to impose the Lorentz invariance of the Lagrangian to obtain any more information.

Since $\xi$ is spinor, we have the freedom to pick a left chiral spinor, so we can have
\begin{equation}
    \delta \phi = \xi \cdot \chi
    \label{eqn:5:boson transformation}
\end{equation}
which is fully Lorentz invariant and a valid term in a Lagrangian.

We move on to the transformation of the fermion.
\begin{equation}
    \delta \chi = -i (\partial\phi) \sigma^\mu (i\sigma^2) \xi^*
    \label{eqn:5:fermion transformation}
\end{equation}
where we obtained this the same way, by imposing the equality of dimensions on both sides of the equation, Lorentz invariablity, and the reality of the Lagrangian.


\chapter{SUSY Charges}
\label{ch:6}

We will start off with a short review on the necessary elements we need to know about symmetry chargers. We will look at how to derive and interpret the charges that make up supersymmetry. Once we have the charges, we will attempt to obtain the supersymmetry transformation rules of fermions, bosons, and auxiliary fields.

\section{Quick review}
We define a unitary transformation as $U \equiv \exp[\pm i \varepsilon\cdot Q]$ where $\varepsilon$ is an infinitesimal factor and $Q$ is the charge of the symmetry. The dot product implies that there may be more than 1 charge to the symmetry. The transformation of a state in the symmetry is defined as $\phi'(x) \equiv U \phi(x) U^\dagger$

The transformation of a state in the symmetry may be defined in 2 manners: through the unitary transformations; or through a varaince.
\begin{eqnarray}
    \phi'(x) &\equiv& U \phi(x) U^\dagger \nonext
    &\equiv& \phi(x) + \delta\phi(x) \nonumber
\end{eqnarray}

Making use of the fact that $\varepsilon$ is infinitesimal, we may, to leading order of $\varepsilon$, relate the variance of $\phi(x)$ to the commutator relation of $Q$ and $\phi$.
\begin{equation}
    \delta \phi(x) = \pm i [\varepsilon\cdot Q, \phi(x)]
    \label{eqn:6:charge and commutator relation}
\end{equation}

The explicit representations of the charges may be obtained from either of 2 way: through the symmetry currents in Equation \ref{eqn:6:charge as symmetry current}; or as differential operators in Equation \ref{eqn:6:charge as differential operator}. We will obtain the algebra of the symmetry if we work out all the commutator relations of the charges in the symmetry.

\begin{equation}
    Q^i = \int \td^3 x J_0^i(\Vec{x}, t)
    \label{eqn:6:charge as symmetry current}
\end{equation}

\begin{equation}
    \phi(x') \equiv \exp[\pm i \varepsilon\cdot\hat{Q}]\phi(x)
    \label{eqn:6:charge as differential operator}
\end{equation}
where here we emphasise $\hat{Q}$ is a differential operator through the hat notation.

Instead of finding the explicit representations of the charges to determine the algebra of the symmetry, we may consider the charges as quantum field operators and work it out. Consider 2 consecutive transformations, with infinitesimal factors $\alpha$ and $\beta$:
\begin{eqnarray}
    U_\beta U_\alpha \phi U_\alpha^\dagger U_\beta^\dagger &\approx& \phi + i [\alpha\cdot Q, \phi] + i [\beta\cdot Q, \phi] - [\beta\cdot Q, [\alpha\cdot Q, \phi]] + ... \nonext
    &=& \delta_\beta \delta_\alpha \phi
\end{eqnarray}

Working out the opposite order,
\begin{eqnarray}
    [\delta_\beta, \delta_\alpha] \phi = \big[[\alpha\cdot Q, \beta\cdot Q], \phi\big] 
    \label{eqn:6:charge as quantum field operators}
\end{eqnarray}

\section{Deriving the supersymmetric charges}
We have 2 charges to consider, since there are 4 degrees of freedom that are be grouped as 2 pairs of spinors (i.e. $\xi$, $\xi^*$). Using the transformation rules from the previous chapter, we can now express them in terms of the SUSY charge commutator relations as
\begin{eqnarray}
    \left[i Q \cdot \xi + i \Bar{Q} \cdot \Bar{\xi} \right] &=& - i \xi \cdot \chi \\
    \label{eqn:6:boson transformation rule full}
    \left[i Q \cdot \xi + i \Bar{Q} \cdot \Bar{\xi} \right] &=& - i (\partial_\mu \phi) \sigma^\mu \sigma^2 \xi^*
    \label{eqn:6:fermion transformation rule full}
\end{eqnarray}

Since $\xi$ and $\xi^*$ are independent, we see that the only non-vanishing terms are:
\begin{eqnarray}
    \left[ \xi\cdot Q, \phi\right] &=& -i \xi \cdot \chi \\
    \label{eqn:6:boson transformation rule}
    \left[\Bar{\xi}\cdot\Bar{Q}, \chi\right] &=& -i (\partial_\mu \phi) \sigma^\mu \sigma^2 \xi^*
    \label{eqn:6:fermion transformation rule}
\end{eqnarray}

Matching the charges with each other in a commutator relation, we get the following:
\begin{eqnarray}
    \left[Q\cdot\xi,Q\cdot\beta\right] &=& (\sigma^2)^{ab}(\sigma^2)^{cd} \xi_b \beta_d \{Q_a, Q_c\} \nonext
    \left[Q\cdot\xi, \Bar{Q}\cdot\Bar{\beta}\right] &=& - (\sigma^2)^{ab}(\sigma^2)^{cd} \xi_b \beta^*_d \{Q_a, Q^\dagger_c\} \nonext
    \left[\Bar{Q}\cdot\Bar{\xi}, Q\cdot\beta\right] &=& (\sigma^2)^{ab}(\sigma^2)^{cd} \xi^*_b \beta_d \{Q^\dagger_a, Q_c\} \nonext
    \left[\Bar{Q}\cdot\Bar{\xi}, \Bar{Q}\cdot\Bar{\beta}\right] &=& (\sigma^2)^{ab}(\sigma^2)^{cd} \xi^*_b \beta^*_d \{Q^\dagger_a, Q^\dagger_c\} \label{eqn:6:SUSY charges commutator relation}
\end{eqnarray}
The algebra of the symmetry is embedded in the anti-commutator relations in these equations.

With these, we know how to get
\begin{equation}
    [\delta_\beta, \delta_\xi] \phi = \big[ \left[Q \cdot \xi + \Bar{Q} \cdot \Bar{\xi}, Q \cdot \beta + \Bar{Q} \cdot \Bar{\beta}\right], \phi \big] \equiv [\mathcal{O}, \phi]
    \label{eqn:6:SUSY transformation commutator relation boson}
\end{equation}

Expanding the LHS of Equation \ref{eqn:6:SUSY transformation commutator relation boson} for a boson,
\begin{eqnarray}
    \left[\delta_\beta, \delta_\xi\right] \phi &=& -i (\xi^\dagger \Bar{\sigma}^\mu \beta - \beta^\dagger \xi) \partial_\mu \phi \nonext
    &=& (\xi^T \sigma^2 \sigma^\mu \sigma^2 \beta^*  - \beta^T \sigma^2 \sigma^\mu \sigma^2 \xi^*) [P_\mu, \phi] \nonext
    \therefore \mathcal{O} &=& (\xi^T \sigma^2 \sigma^\mu \sigma^2 \beta^*  - \beta^T \sigma^2 \sigma^\mu \sigma^2 \xi^*) P_\mu \nonext
    &=& -(\sigma^2)^{ab} (\sigma^2)^{cd} (\xi_b \beta^*_d \sigma^\mu_ac + \xi^*_b \beta_d \sigma^\mu_{ca}) P_\mu
\end{eqnarray}

Comparing against the coefficients in Equations \ref{eqn:6:charge and commutator relation}, we will arrive at the following anti-commutator relations:
\begin{eqnarray}
    \{Q_a, Q_c\} &=& \{Q_a^\dagger, Q_c^\dagger\} = 0
    \nonext
    \{Q_a, Q_c^\dagger\} &=& \sigma^\mu_{ac} P_\mu\nonext
    \{Q_a^\dagger, Q_c\} &=& \sigma^\mu_{ca} P_\mu\nonumber
    \label{eqn:6:susy algebra}
\end{eqnarray}
and by normalising the charges, $Q \rightarrow Q/\sqrt{2}$, the non-vanishing anti-commutator relations are
\begin{eqnarray}
    \{Q_a, Q_c^\dagger\} &=& 2 \sigma^\mu_{ac} P_\mu \\
    \label{eqn:6:susy algebra 1}
    \{Q_a^\dagger, Q_c\} &=& 2 \sigma^\mu_{ca} P_\mu 
    \label{eqn:6:susy algebra 2}
\end{eqnarray}

Note that since $Q$ and $\Bar{Q}$ are spacetime independent, the algebra between the momentum Poincar\'{e} charges and supersymmetric charges necessarily vanish.
\begin{equation}
    [Q, P_\mu] = [Q^\dagger, P_\mu] = 0
\end{equation}

However, the angular Poincae\'{e} charges and supersymmetric charges do not vanish. 
\begin{equation}
    [Q_a, M_{\mu\nu}] = (\sigma_{\mu\nu})^b_a Q_b \quad , \quad \sigma_{\mu\nu} \equiv \frac{i}{4} (\sigma_\mu \Bar{\sigma}_\nu - \sigma_\nu \Bar{\sigma}_\mu)
\end{equation}

If we were to conclude that the algebra for SUSY is complete with this, we would be sorely mistaken as it does not close for the spinor fields as they are now. This is simply because the spinor fields we have are on-shell spinors with a total of 2 degrees of freedom, 2 short of the bosonic degrees of freedom. To handle this, we will have to introduce auxiliary fields that will vanish on-shell while accounting for the missing 2 degrees of freedom. We will allow the auxiliary fields to be bosonic. Since they must vanish on-shell, the simplest form they can take is $F^\dagger F$. Naturally, the dimension for the auxiliary field has to be 2 in the Lagrangians we have been working in. The free field Lagrangian is now:

\begin{equation}
    \mathcal{L} = \partial_\mu \phi^\dagger \partial^\mu \phi + \chi^\dagger i \Bar{\sigma}^\mu \partial_\mu \chi + F^\dagger F
    \label{eqn:6:new lagrangian}
\end{equation}

The explicit transformation rule of $F$ needs to be linear in the infinitesimal $\xi$ and one other field, all while ensuring its dimension and Lorentz invariance. The right choice of $\delta F$ is
\begin{equation}
    \delta F = K \xi^\dagger \Bar{\sigma}^\mu\partial_\mu\chi
\end{equation}

To ensure that this addition of the auxiliary field to the Lagrangian will not interfere with the overall invariance, we have to apply a variance on the fields.
\begin{equation}
    \delta(F^\dagger F) = (K^* \xi F)^\dagger \Bar{\sigma}^\mu \partial_\mu \chi - \chi^\dagger \Bar{\sigma}^\mu \partial_\mu (K^* \xi F)
    \label{eqn:6:auxiliary fields variance}
\end{equation}
Noticing that this is similar in structure to the variance of the free spinor fields in Equation \ref{eqn:6:free spinor fields variance}, we can define a new spinor field as in Equation \ref{eqn:6:new spinor field}.
\begin{equation}
    \delta(\chi^\dagger i \Bar{\sigma}^\mu \partial_\mu \chi) = (\delta \chi)^\dagger i \Bar{\sigma}^\mu \partial_\mu \chi + \chi^\dagger i \Bar{\sigma}^\mu \partial_\mu (\delta\chi)
    \label{eqn:6:free spinor fields variance}
\end{equation}

\begin{equation}
    \delta \Tilde{\chi} \equiv \delta \chi - i K^* \xi F
    \label{eqn:6:new spinor field}
\end{equation}

Because of the freedom we have for K, we can conveniently set it to $i$. This way, our new Lagrangian in Equation \ref{eqn:6:new lagrangian} will be closed under the SUSY algebra in Equations \ref{eqn:6:susy algebra 1} and \ref{eqn:6:susy algebra 2} with the following field super-transformations:
\begin{eqnarray}
    \delta \phi &=& \xi \cdot\chi \nonext
    \delta \chi &=& - i \sigma^\mu (i \sigma^2 \xi^*) \partial_\mu \phi + F\xi \label{eqn:6:supersymmetric field transformations} \\
    \delta F &=& -i \xi^\dagger \Bar{\sigma^\mu} \partial_\mu \chi \nonumber
\end{eqnarray}

\chapter{Applications of SUSY algebra}
\label{ch:7}
Continuing off the previous chapter, we will work on the SUSY algebra to create the SUSY multiplets. We will also formulate the algebra again, but in the Majorana form to see the significance of its interpretation over the Weyl spinor representation. We will then attempt to find the explicit forms of the supercharges both as symmetry currents and quantum filed operators. Lastly, we will discuss about the extension of the algebra into areas outside of SUSY.

\section{Casimir Operators}
\label{ch:7:casimir operator}
The Casimir operators are operators which commute with every generator of a group. Because of that, its eigenvalues can be used to classify the group representations. For example, $P^\mu P_\mu$ is a Casimir operator of the Poincar\'{e} group with an eigenvalue $m^2$. Another (more useful) Casimir operator of the Poincar\'{e} group is the Pauli-Lubonski operator $W^\mu$.
\begin{equation}
    W^\mu \equiv \frac{1}{2} \varepsilon^{\mu \nu \sigma \rho} M_{\rho\sigma} P_\nu
\end{equation}

For a massive particle, the Pauli-Lubonski operator gives the total angular momentum of a particle as its eigenvalue.
\begin{equation}
    W^i\ket{p} = m(L^i + S^i)\ket{p}
\end{equation}

Contracting it with itself and applying onto a particle at rest,
\begin{equation}
    W_\mu W^\mu \ket{p} = -m^3 s(s+1) \ket{p}
\end{equation}

On the other hand, on a massless particle, the eigenvalue of the Pauli-Lubonski operator is the helicity of the particle. Take for example a massless particle in with an angle of rotation in the z-axis (i.e. $P^\mu \ket{p} = (E, 0, 0, E) \ket{p}$).
\begin{equation}
    W^\mu \ket{p}  = (Es_z, 0, 0, Es_z) \ket{p}
\end{equation}

\section{Applying onto supercharges}
\label{ch:7:applying onto supercharges}
Applying the Pauli-Lubonski operator to the supercharges in a commutator relation,
\begin{equation}
    [Q_a, W^0] = - \frac{1}{2} (\sigma^3)^b_a Q_b P_3
\end{equation}
The only non-zero terms of $\sigma^3$ are the diagonal terms so what we essentially have is:
\begin{eqnarray}
    \left[ Q_1, W_0\right] &=& - \frac{1}{2}Q_1 P_3 \\
    \left[ Q_2, W_0\right] &=& \frac{1}{2}Q_2 P_3
\end{eqnarray}  

We can use this to derive what the supercharges do onto a particle state.
\begin{eqnarray}
    W_0(Q_1\ket{p, h}) &=& [W_0, Q_1] \ket{p,h} + Q_1W_0\ket{p,h} \nonext
    &=& E(h+\frac{1}{2}) Q_1 \ket{p,h} \nonext
    \implies Q_1 \ket{p,h} &=& \ket{p, h+\frac{1}{2}}
    \label{eqn:7:q1 supercharge}
\end{eqnarray}
Likewise for $Q_2$:
\begin{equation}
    Q_2 \ket{p,h} = \ket{p, h-\frac{1}{2}}
    \label{eqn:7:q2 supercharge}
\end{equation}

From Equations \ref{eqn:7:q1 supercharge} and \ref{eqn:7:q2 supercharge}, we now know that $Q_1$ raises the helicity of the particle by $\frac{1}{2}$ whereas $Q_2$ lowers the helicity of the particle by $\frac{1}{2}$.

\section{Building the SUSY multiplets}
Let us look at the massless, rest particle. Recalling the SUSY algebra in Equation \ref{eqn:6:susy algebra}, the algebra all depend on $\sigma^\mu_{ab}P_\mu$.
\begin{equation}
    \sigma^\mu_{ab} P_\mu =
    \begin{pmatrix}
        0 & 0 \\ 0 & 2P^0
    \end{pmatrix} \ket{p,h}
\end{equation}

Thus, the only non-vanishing algebra is
\begin{equation}
    \{Q_2, Q_2^\dagger\} = 2 E \ket{p,h}
\end{equation}

The vanishing algebra also sheds some insight into the inner workings of the multiplet.
\begin{equation}
    \{Q_1, Q_1^\dagger\} = 0 \implies Q_1\ket{p,h} = Q_1^\dagger \ket{p,h} = 0
\end{equation}
This means that in SUSY multiplet, there is a minimum helicity to consider!

Let us define the minimum helicity $h_{min}$
\begin{equation}
    Q_2\ket{p, h_{min}} = 0 \quad , \quad Q_2^\dagger\ket{p, h_{min}} = \ket{p, h_{min} + \frac{1}{2}}
\end{equation}

Moreover,
\begin{equation}
    \{Q_2^\dagger, Q_2^\dagger\} = 0 \implies Q_2^\dagger Q_2^\dagger \ket{p, h_{min}} = 0
\end{equation}
this implies that the multiplet has only 2 states of the same momentum, but a helical difference of $1/2$.

To make the duet CPT invariant, we need to add the CPT conjugates of each of the 2 states. Thus, there needs to be 4 states to a SUSY multiplet. For example, if $h_{min} = 0$, we will have a scalar multiplet with helicities $0, 0, \frac{1}{2}, -\frac{1}{2}$; if $h_{min} = \frac{1}{2}$, we will have a vector multiplet with helicities $\frac{1}{2}, 1, -\frac{1}{2}, -1$.

\section{Supercharges through symmetry currents}
\label{ch:7:supercharges through symmetry currents}

The general Lagrangian made of complex scalar fields is:
\begin{equation}
    \mathcal{L} = \mathcal{L}(\phi, \phi^\dagger, \partial_\mu \phi, \partial_\mu \phi ^\dagger)
\end{equation}

The variance of the Lagrangian is:
\begin{equation}
    \delta \mathcal{L} = \diff{\mathcal{L}}{\phi} \delta \phi + \diff{\mathcal{L}}{\phi^\dagger} \delta \phi^\dagger + \diff{\mathcal{L}}{(\partial_\mu \phi)} \delta (\partial_\mu \phi) + \diff{\mathcal{L}}{(\partial_\mu \phi^\dagger)}
\end{equation}
where on-shell, 
\begin{equation}
    \diff{\mathcal{L}}{\phi} = \partial_\mu \diff{\mathcal{L}}{\partial(\partial_\mu \phi)}
\end{equation}
giving us
\begin{equation}
    \partial_\mu \mathcal{K^\mu} \equiv \delta \mathcal{L} = \partial_\mu \left[ \diff{\mathcal{L}}{(\partial_\mu \phi)} \delta \phi + \diff{\mathcal{L}}{(\partial_\mu \phi^\dagger)} \delta \phi^\dagger \right]
\end{equation}
where $\mathcal{K^\mu}$ is introduced as we know that in general, the variance of the Lagrangian can total differential as it will disappear in the integral. The terms in the bracket are the Noether's current, denoted by $j^\mu$. The conserved current $J^\mu$ is thus defined as:
\begin{equation}
    J^\mu = j^\mu - K^\mu
    \label{eqn:7:symmetry current}
\end{equation}

The supercharges are derived using the 0-th index of $J^\mu$ as in Equation \ref{eqn:6:charge as symmetry current}.

\section{VEV of the Hamiltonian}
\label{ch:7:vev of hamiltonian}
From
\begin{equation}
    \{Q_a, Q_b^\dagger\} = \sigma^\mu P_\mu \implies \braket{\{Q_1, Q_1^\dagger\} + \{Q_2, Q_2^\dagger\}} = 2 \braket{\mathcal{H}}
\end{equation}

The positive-definitivity of the LHS implies that $\braket{\mathcal{H}} \geq 0$. The equality is achieved when both supercharges annihilate the vacuum state, and by extension, a strict inequality is enforced when the supercharges do not annihiliate the vacuum state -- spontaneous supersymmetry breaking.

\section{SUSY in the Majorana Form}
\label{ch:7:susy in majorana form}

Recall that the right chiral spinor of the Majorana spinor is related to its left chiral spinor as $\eta = i \sigma^2 {\chi^\dagger}^T$. This means that all 4 components of the Majorana supercharge can be expressed as
\begin{eqnarray}
    Q_M &\equiv& \col{i \sigma^2 {Q^\dagger}^T}{Q} = \begin{pmatrix} -Q_2^\dagger \\ -Q_1^\dagger \\ Q_1 \\ Q_2 \end{pmatrix}
\end{eqnarray}

\section{Explicit supercharges}
\label{ch:7:explicit supercharges}
In Equations \ref{eqn:6:charge as symmetry current}, we saw how to obtain the supercharges explicitly using the conserved symmetry current. To get the expression for the conserved symmetry current, we need the Noether's current ($j^\mu$) and the surface differential terms that might have been `discarded' in the derivation of the Lagrangian ($\partial_\mu K^\mu$). 

For example, in the free supersymmetric Lagrangian
\begin{equation}
    \mathcal{L} = \partial_\mu \phi^\dagger \partial^\mu \phi + \chi^\dagger i \Bar{\sigma}^\mu \partial_\mu \chi + F^\dagger F
    \label{eqn:7:free lagrangian}
\end{equation}
its conserved current is
\begin{equation}
    \mathcal{J}^\mu_{SUSY} = (\partial_\nu \phi) \chi^\dagger \Bar{\sigma}^\mu \sigma^\nu (i\sigma^2) \xi^* - (\partial_\nu \phi^\dagger) \xi^T (i\sigma^2) \sigma^\nu \Bar{\sigma}^\mu \chi
    \label{eqn:7:conserved current}
\end{equation}

Putting Equation \ref{eqn:7:conserved current} into Equation \ref{eqn:6:charge as symmetry current}, we get
\begin{equation}
    \xi\cdot Q + \Bar{\xi} \cdot \Bar{Q} = \int \td^3 x (\partial_\nu \phi) \chi^\dagger \Bar{\sigma}^\mu \sigma^\nu (i\sigma^2) \xi^* - (\partial_\nu \phi^\dagger) \xi^T (i\sigma^2) \sigma^\nu \Bar{\sigma}^\mu \chi
\end{equation}

Comparing the coefficients of $\xi$ and $\xi^*$,
\begin{eqnarray}
    Q = \int \td^3 x \partial_\nu \phi^\dagger \sigma^\nu \Bar{\sigma}\mu \chi \\
    Q^\dagger = \int \td^3 x \chi^\dagger \Bar{\sigma}^\mu \sigma^\nu \partial_\nu \phi^\dagger
\end{eqnarray}

Two more identities that need to be included when using the explicit charges are
\begin{eqnarray}
    \left[ \phi(x,t), \Dot{\phi}^\dagger(y,t)\right] &=& i \delta^3(x-y) \\
    \left\{ \chi_a(x,t) , \chi_b^\dagger(y,t)\right\} &=& \delta_{ab} \delta^3(x-y)
\end{eqnarray}

With these 4 equations, the explicit supercharges may be used freely in applications such as verifying the field transformations.
\chapter{The Wess-Zumino Model}
\label{ch:8}

Up to this point, we have explicitly formulated the Lagrangian of free particles. We realised that for the SUSY algebra to be complete for both the boson and spinor fields, we needed to introduce a new field that disappears on-shell. In this chapter, we will look into adding some interactions between the fields so as to bring our discussions away from the toy model that it is right now. We will interactions between all 3 fields and their hermitian conjugates. Once our new Lagrangian is complete, we will be able to employ some tricks that is used widely in Lagrangian mechanics and identify an umbrella potential term that will be useful in our future attempt to break the symmetry. Lastly, as we did for the past few chapters, we will express the Lagrangian in the Majorana spinor representation.

\section{Interactions to consider}
\label{ch:8:interactions to consider}
We now have a total of 6 fields: $\phi, \phi^\dagger, F, F^\dagger, \chi, \chi^\dagger$. We will only be considering interactions that will satisfy the following constraints in the Lagrangian:
\begin{enumerate}
    \item $D\leq4$
    \item Lorentz invariance
    \item Hermitivity
    \item Gauge invariance
\end{enumerate}

We can exclude the 4th option for now since we are working without a gauge charge in this model, but in general these are the constraints in picking possible interactions for a super-renormalisable theory.

\subsection{$\phi$ and $\phi^\dagger$ interactions only}
Any arbitrary function $G(\phi, \phi^\dagger)$ will satisfy all 3 conditions.

\subsection{$\phi$, $\phi^\dagger$, $F$, and $F^\dagger$ interactions}
A function with the form $W_1(\phi, \phi^\dagger) F + h.c.$ will satisfy all 3 conditions. Note that $W_1$ is at most quadratic or bilinear in terms.

\subsection{$\phi$, $\phi^\dagger$, $\chi$, and $\chi^\dagger$ interactions}
To get Lorentz invariant terms of the spinors, we make use of what we have done in Chapter \ref{ch:2}. $\chi\cdot\chi$ and $\Bar{\chi}\cdot\Bar{\chi}$ are Lorentz invariants. Possible interactions of this group come in the form $-\frac{1}{2} W_{11} (\phi, \phi^\dagger) \chi\cdot\chi + h.c.$. Note that $W_{11}$ is at most linear in terms.

\subsection{$\chi$, $\chi^\dagger$, $F$, and $F^\dagger$ interactions}
Any Lorentz invariant combinations of these terms will necessarily violate the first condtion, and thus we need no consider any of these interactions for the Wess-Zumino model.

The indices of $W_1$ and $W_{11}$ do not make sense now, but by the end we will see that they are indices of the fields they are attached to -- $F_i$ and $\chi_i\cdot\chi_j$.

\section{The General Wess-Zumino Lagrangian}
Let us put in the interactions we have guessed in the previous section.
\begin{equation}
    \mathcal{L}_{WZ} = \partial_\mu \phi^\dagger \partial^\mu\phi + \chi^\dagger i \Bar{\sigma}^\mu \partial_\mu \chi + F^\dagger F + G + W_1 F + W_1^\dagger F^\dagger - \frac{1}{2} W_{11} \chi\cdot\chi - \frac{1}{2}W_{11}^\dagger \Bar{\chi}\cdot\Bar{\chi}
    \label{eqn:8:general wess-zumino lagrangian}
\end{equation}

We now need $\mathcal{L}_{int}$ to transform supersymmetrically, as $\mathcal{L}_{free}$ did. We will do this by varying $\mathcal{L}_{int}$ and using the supersymmetric transformation rules as in Equations \ref{eqn:6:supersymmetric field transformations}, ensure that the total variance either vanishes or gets swept away as a total derivative.

Doing the work, we will see that
\begin{eqnarray}
    \delta\mathcal{L}_{int} &=& \diff{G}{\phi}\chi\cdot\xi + \diff{W_1}{\phi}\chi\cdot\xi F + \diff{W_1}{\phi^\dagger}\Bar{\chi}\cdot\Bar{\xi}F - i W_1 \xi^\dagger \Bar{\sigma}^\mu \partial_\mu \chi \nonext
    && - \frac{1}{2} \diff{W_{11}}{\phi} \chi \cdot\xi \chi\cdot\chi - \frac{1}{2}\diff{W_{11}}{\phi^\dagger} \Bar{\chi}\cdot\Bar{\xi} \chi \cdot \chi \nonext
    && - i W_{11} \chi^T i \sigma^2 \partial_\mu \phi \sigma^\mu i \sigma^2 \xi^* - W_{11} F \chi \cdot \xi + h.c.
    \label{eqn:8:wess-zumino lagrangian variance}
\end{eqnarray}

We see that for this to either vanish or gauge away as a total derivative, coefficients of the combinations of fields need to either vanish or gauge away as a total derivative. This gives us the following constraints:
\begin{equation}
    \chi \cdot \xi \neq 0 \implies \diff{G}{\phi} = 0
    \label{eqn:8:G diff}
\end{equation}
\begin{equation}
    \Bar{\chi}\cdot\Bar{\xi} \chi\cdot\chi\neq 0 \implies \diff{W_{11}}{\phi^\dagger} = 0
    \label{eqn:8:W_11 phi dagger diff}
\end{equation}
\begin{equation}
    \Bar{\chi}\cdot\Bar{\xi} F \neq 0 \implies \diff{W_1}{\phi^\dagger} = 0
    \label{eqn:8:W_1 phi dagger diff}
\end{equation}
\begin{equation}
    \chi\cdot\xi\chi\cdot\chi = 0 \implies \diff{W_{11}}{\phi} \text{ has no constraints}
    \label{eqn:8:W_11 phi diff}
\end{equation}
\begin{equation}
    \chi \cdot \xi F \neq 0 \implies \diff{W_1}{\phi} - W_{11} = 0
    \label{eqn:8:W_1 and W_11 relation}
\end{equation}
\begin{equation}
    \chi^T (i \sigma^2) \sigma^\mu (i \sigma^2) \xi^* \neq 0 \implies \partial_\mu W_1 = W_{11} \partial_\mu \phi
    \label{eqn:8:W_1 and W_11 relation partial}
\end{equation}
and all their hermitian conjugates.

We have to note that $G$ is real, so Equation \ref{eqn:8:G diff} tells us that $G$ is a constant that we can conveniently set to 0. Equations \ref{eqn:8:W_11 phi dagger diff} and \ref{eqn:8:W_1 phi dagger diff} tells us that $W_{1}$ and $W_{11}$ are holomorphic in $\phi$ (i.e. $W_1 = W_1(\phi)$ and $W_{11} = W_{11}(\phi)$). The final 2 equations, Equations \ref{eqn:8:W_1 and W_11 relation} and \ref{eqn:8:W_1 and W_11 relation partial} are identical to each other -- $W_{11} = \diff{W_1}{\phi}$. Equation \ref{eqn:8:W_11 phi diff} tells us that $W_{11}$ is the only degree of freedom we have.

With these, we can now write the interaction Lagrangian as
\begin{equation}
    \mathcal{L}_{int} = W_1(\phi) F - \frac{1}{2} \diff{W_1}{\phi} \chi\cdot\chi + h.c.
\end{equation}

To have this Lagrangian satisfy $D \leq 4$, $[W_1] \leq 2$, which means that $W_1$ is at most quadratic in $\phi$. Its most general form is
\begin{equation}
    W_1 = m \phi + \frac{1}{2} y \phi^2 + C
\end{equation}
where $y$ is a dimensionless constant.

Here, we take a page off classical Lagrangian mechanics. The interaction Lagrangian may be written a a derivative the derivative of a potential term, for which we will label as $\mathcal{W}$.
\begin{equation}
    \mathcal{W} = \frac{1}{2} m \phi^2 + \frac{1}{6}y \phi^3 + C\phi + f(\phi^\dagger)
\end{equation}

The notation so far is complete for the 1 particle Wess-Zumino interaction, where we need not worry about particle indices and can leave all the indices at $1$ or $11$. If we were to extend this to $n$-particles however, we have to make a slight correction.

\begin{eqnarray}
    \mathcal{L}_{WZ} &=& \sum_{i} \partial_\mu \phi^\dagger_i \partial^\mu \phi_i + \chi_i^\dagger i \Bar{\sigma}^\mu \partial_\mu \chi_i + F_i^\dagger F_i \nonext
    &&+ \left(\sum_{i,j} W_i F_i - \frac{1}{2} W_{ij}\chi_i\cdot\chi_j + h.c.\right)
    \label{eqn:8:wess-zumino lagrangian for n particles}
\end{eqnarray}

Going through the process again, we will see that $W_{ij}$ has to be at most linear in $\phi$. Its most general form has to be
\begin{equation}
    W_{ij} = m_{ij} + y_{ijk}\phi_k
\end{equation}
and for us to arrive at this, we had to impose that $\diff{W_{ij}}{\phi_k}$ is cyclic invariant, which then implies that $y_{ijk}$ also needs to have cyclic symmetry. $\chi_i\cdot\chi_j$ is symmetric in indices so $m_{ij}$ also has to be symmetric in indices. Therefore, we have found that $W_{ij}$ is completely symmetric. One simply way to ensure its symmetricity is to have $W_{ij}$ be a second order differential of a function.
\begin{equation}
    W_{ij} = \diff{^2\mathcal{W}}{\phi_i \partial\phi_j}
\end{equation}

With the other constraints that $W_i$ also having to be holomorphic in $\phi_i$ and that $W_i = \diff{\mathcal{W}}{\phi_i}$, the most general form $\mathcal{L}$ can take is:
\begin{equation}
    \mathcal{W} = \frac{1}{2}m_{ij}\phi_i \phi_j + \frac{1}{6}y_{ijk}\phi_i\phi_j\phi_k + c_i \phi_i
\end{equation}

Now, we can organise Equation \ref{eqn:8:wess-zumino lagrangian for n particles} in terms of the superpotential $\mathcal{W}$ so that it would be easier to see the physics.
\begin{eqnarray}
    \mathcal{L}_{WZ} &=& \sum_{i} \partial_\mu \phi^\dagger_i \partial^\mu \phi_i + \chi_i^\dagger i \Bar{\sigma}^\mu \partial_\mu \chi_i + F_i^\dagger F_i \nonext
    &&+ \left(\diff{\mathcal{W}}{\phi_i} W_i F_i + \frac{1}{2} \diff{^2\mathcal{W}}{\phi_i\partial\phi_j}\chi_i\cdot\chi_j + h.c.\right)
    \label{eqn:8:wess-zumino lagrangian for n particles with superpotential}
\end{eqnarray}

Our Wess-Zumino Lagrangian is now completely supersymmetric! We can work a little more to remove the auxiliary fields from the Lagrangian solving their equations of motion (which we now for a fact is vanishing).
\begin{equation}
    F_i^\dagger F_i^\dagger = -\left\vert\diff{\mathcal{W}}{\phi_i}\right\vert^2
\end{equation}

\begin{equation}
    \therefore \mathcal{L}_{WZ} = \partial_\mu \phi^\dagger_i \partial^\mu \phi_i + \chi_i^\dagger i \Bar{\sigma}^\mu \partial_\mu \chi_i - \left\vert\diff{\mathcal{W}}{\phi_i}\right\vert^2 - \frac{1}{2}\left(\diff{^2\mathcal{W}}{\phi_i \partial_j}\chi_i\cdot\chi_j + h.c.\right)
    \label{eqn:8:wess-zumino lagrangian for n particles without auxiliary fields}
\end{equation}
This form is a much more insightful than Equation \ref{eqn:8:wess-zumino lagrangian for n particles with superpotential} as here, all our terms are built off the physical boson and spinor fields. It also shows us exactly where the potential terms are in the Lagrangian as it comes very neatly in the form $\mathcal{L} = T - V$.

Being explicit in Equation \ref{eqn:8:wess-zumino lagrangian for n particles without auxiliary fields}, the Lagrangian becomes
\begin{eqnarray}
    \mathcal{L}_{WZ} &=& \partial_\mu \phi_i^\dagger \partial^\mu \phi_i + \chi_i^\dagger i \Bar{\sigma}^\mu \partial_\mu \chi_i - \left\vert m_{ij} \phi_j + \frac{1}{2} y_{ijk}\phi_j\phi_k + c_i\right\vert^2 \nonext
    && - \frac{1}{2} (m_{ij} \chi_i\cdot\chi_j + y_{ijk}\phi_k\chi_i\cdot\chi_j + h.c. )
    \label{eqn:8:wess-zumino lagrangian expanded}
\end{eqnarray}
and we see that the masses of the bosons and spinors are the same!

\section{The Wess-Zumino Lagrangian in Majorana Form}
\label{ch:8:wess-zumino lagrangian in majorana form}
Let us consider a single particle and set $c = 0$ Equation \ref{eqn:8:wess-zumino lagrangian expanded} becomes 
\begin{eqnarray}
    \mathcal{L}_{WZ} &=& \mathcal{L}_M + \partial_\mu \phi^\dagger \partial^\mu \phi - m^2 \phi^\dagger \phi - \frac{1}{2}my({\phi^\dagger}^2\phi + \phi^\dagger\phi^2) \nonext 
    &&- \frac{1}{4}y^2(\phi^\dagger\phi)^2 - \frac{1}{2}y(\phi\chi\cdot\chi + \phi^\dagger\Bar{\chi}\cdot\Bar{\chi})
    \label{eqn:8:wess-zumino lagrangian with majorana lagrangian}
\end{eqnarray}

We can decompose the complex scalar $\phi$ into its components $\phi = \frac{1}{\sqrt{2}}(A + i B)$ and express Equation \ref{eqn:8:wess-zumino lagrangian with majorana lagrangian} in terms of $A$, $B$, and $\Psi_M$. We will define $g \equiv \frac{1}{\sqrt{8}}y$ and the following terms will then be:
\begin{equation}
    -\frac{1}{4}y^2(\phi^\dagger \phi)^2 = -\frac{1}{2} g^2 (A^2 + B^2)^2 \equiv \mathcal{L}_1
    \label{eqn:8:L1}
\end{equation}
\begin{equation}
    -\frac{1}{2}my{\phi^\dagger}^2\phi + h.c. = - mg (A^3 + AB^2) \equiv \mathcal{L}_2
    \label{eqn:8:L2}
\end{equation}
\begin{equation}
    -\frac{1}{2}y(\phi\chi\cdot\chi + \phi^\dagger \Bar{\chi}\cdot\Bar{\chi}) = -g(A\Bar{\Psi}_M\Psi_M + iB\Bar{\Psi}_M\gamma^5\Psi_M) \equiv \mathcal{L}_3 + \mathcal{L}_4
\end{equation}

We now have
\begin{equation}
    \mathcal{L}_{WZ} = \mathcal{L}_{Free, WZ} + \mathcal{L}_1 + \mathcal{L}_2 + \mathcal{L}_3 + \mathcal{L}_4
\end{equation}

where the numbered Lagrangians will come in very handy when we do diagram calculations in the next chapter.
\chapter{Some explicit calculations}
\label{ch:9}

Here, we will carry out some explicit calculations on the Wess-Zumino model and explicitly show the most desirable trait of supersymmetry -- the vanishing quadratic divergences. The calculations carried out will be diagram calculations in QFT and will be briefly covered in the first section. Afterwards, we will apply those onto the interacting Lagrangians in Chapter \ref{ch:8} Section \ref{ch:8:wess-zumino lagrangian in majorana form}. Lastly, we will discuss the renormalisability of the theory to handle the logarithmic divergences that do not vanish like the quadratic divergences. 

\section{Quick overview of QFT process calculation}
\label{ch:9:quick overview of qft process calculation}
We will be working the calculations on n-point functions. This is where we will have calculations such as
\begin{equation}
    \braket{\Omega|T(\phi(x)\phi(y)..)|\Omega} = \frac{\Braket{0|T\left(\phi_1(x)\phi_2(y)...\exp\left[i\int\td^4z \mathcal{L}_{int}\phi_I(z)\right]\right)|0}}{\Braket{0|T\lc\exp\left[i\int\td^4z \mathcal{L}_{int}\phi_I(z)\right]\rc|0}}
\end{equation}

Several important results are the 2-point functions of scalar fields in the $\lambda\phi^4$ theory:
\begin{equation}
    \Braket{0|\phi(x)\phi(y)|0} = D(x-y) + \mathcal{O}(\lambda)
\end{equation}
where
\begin{eqnarray}
    D(x-y) &=& \int \frac{\td^4 k}{(2\pi)^4} \exp[-ik\cdot(x-y)] \frac{i}{k^2 - m^2 + i\varepsilon} \nonext
    &\equiv& \int \frac{\td^4 k}{(2\pi)^4} \exp[-ik\cdot(x-y)] D(k)
\end{eqnarray}

To order $\lambda$ of the same theory, we have another result:
\begin{eqnarray}
    D_1(x-y) &\equiv& -i \lambda \Braket{0|T\lc\phi(x)\phi(y)\int\td^4z\phi^4(z) \rc|0} \nonext
    &=& -12i \lambda \int \td^4 z \Braket{0|[\phi(x)\sim\phi(z)][\phi(y)\sim\phi(z)][\phi(z)\sim\phi(z)]|0} \nonext
    &=& -12i \lambda \int \td^4 z D(x-z)D(y-z)D(z-z) \label{eqn:9:single loop interaction}\\
    &=& -12i \lambda \int \frac{\td^4 p}{(2\pi)^4} e^{-ip\cdot(x-y)} D(p)D(p)\int\frac{\td^4 q}{(2\pi)^4} \frac{i}{q^2-m^2+i\varepsilon}
    \label{eqn:9:single loop interaction before fourier transformation}
\end{eqnarray}
The diagram of such an interaction visualised from Equation \ref{eqn:9:single loop interaction} is a loop at point $z$ with ends at $x$ and $y$. If we Fourier transform the LHS to the $p$-momentum space, we can match it to Equation \ref{eqn:9:single loop interaction before fourier transformation} and see that
\begin{equation}
    \mathcal{F}\left\{\int\td^4 z D(x-z)D(y-z)D(z-z)\right\} = D(p)D(p)I_d
    \label{eqn:9:single loop interaction after fourier transformation}
\end{equation}
for 
\begin{eqnarray}
    I_d &\equiv& \int_0^\Lambda \frac{\td^4 q}{(2\pi)^4} \frac{i}{q^2 - m^2 + i\varepsilon} \nonext
    &\approx& \frac{1}{8\pi^2} \left[ \Lambda^2 - m^2 \ln\lc\frac{\Lambda}{m}\rc - c\right]
    \label{eqn:9:single loop Id}
\end{eqnarray}
where we see the quadratic and logarithmic terms all together, with a finite $c$. The common technique to work with these is to amputate $D(p)$ from Equation \ref{eqn:9:single loop interaction after fourier transformation} to get the divergences
\begin{equation}
    D_1^{Am}(p) = -12i \lambda I_d
\end{equation}

As for Majorana spinors,
\begin{eqnarray}
    \Braket{0|T\lc\Psi^M_\alpha(x) \Bar{\Psi}^M_\beta(y)\rc|0} &=& \int\frac{\td^4 k}{(2\pi)^4} e^{-ik\cdot(x-y)} S_{\alpha\beta}(k)
\end{eqnarray}
where
\begin{equation}
    S_{\alpha\beta} \equiv i \frac{(\slashed k + m)_{\alpha\beta}}{k^2 - m^2 + i \varepsilon}
\end{equation}

With $(\Psi^M)^C = C \Bar{\Psi}^{M^T}$ and $C^2 = - \mathbb{1}$,
\begin{equation}
    \Braket{0|T\lc\Psi^M_\alpha(x)\Psi^M_\beta(y)\rc|0} = \int\frac{\td^4 k}{(2\pi)^4} e^{-ik\cdot(x-y)} S_{\alpha\beta}(k) C_{\gamma\beta}^T
\end{equation}
\begin{equation}
    \Braket{0|T\lc\bar{\Psi}^M_\alpha(x)\bar{\Psi}^M_\beta(y)\rc|0} = \int\frac{\td^4 k}{(2\pi)^4} e^{-ik\cdot(x-y)} C_{\alpha\gamma}^T S_{\gamma\beta}(k) 
\end{equation}

\section{Explicit calculations on the Wess-Zumino model}
\label{ch:9:explicit calculations on the wess-zumino model}
Now that we have covered the necessary, let us apply them onto the Wess-Zumino model. For the interacting Lagrangian in the exponential of the n-point function, we will use the interacting Lagrangian of the Wess-Zumino model, $\mathcal{L}_{int} = \mathcal{L}_1 + \mathcal{L}_2 + \mathcal{L}_3 + \mathcal{L}_4$. What we are interested in is to see that all the terms vanish except for the logarithmic divergence. For that, we will be grouping the coefficients of of the terms after the calculations and adding them together.

\subsection{Single $A$ field propagator}
The non-vanishing contributions of having an $A$ field particle propagate in the Wess-Zumino model is to the order of $g$. 
\begin{equation}
    \Braket{\Omega|T(A(x))|\Omega} = -ig \int \td^4 z \Braket{0|T\left\{A(x) \lc mA^3(z) + mA(z)B^2(z) + A(z)\bar{\Psi}\Psi \rc\right\}|0}
\end{equation}
Carrying out the calculation (amputating the $D(p)$ terms to make it easier), we would see that the coefficient of $I_d$ for the $A$ field propagator vanishes!

\subsection{Double $B$ field propagator}
There is no contribution with a 1st order of $g$ because every term in $\mathcal{L}_{int}$ to order $g$ has an odd number of $A$ field, and the VEV will naturally vanish. So the terms to consider are the terms with $g$ to the 0th order and 2nd order. 
\begin{eqnarray}
    && \Braket{\Omega|T\lc B(x)B(y)\rc|\Omega} \nonext
    &=& D_B(x-y) - \int\td^4z\int\td^4w \Braket{0|T\lc B(x)B(y) \mathcal{L}_1 \rc|0} \nonext
    && - \frac{1}{2} \int\td^4z\int\td^4w \Braket{0|T\lc B(x)B(y) (\mathcal{L}_2 + \mathcal{L}_3 + \mathcal{L}_4)(z)(\mathcal{L}_2 + \mathcal{L}_3 + \mathcal{L}_4)(w) \rc|0} \nonext
\end{eqnarray}
Carrying out the calculation (and amputating the $D(p)$ terms to make it easier), we yet again see that the coefficients of $I_d$ vanishes! However, there is a remnant term that carries a logarithmic divergence. It is at this point where we should take a step back and allow the flow of things to work on as it can be shown that with a renormalisation of the fields, this logarithmic divergence will too, vanish.

This is the power of SUSY that appeals to the theoretical side of particle physicists. The simple and elegant removal of the ugly divergences that plagues the most successful theory to date.
\chapter{Supersymmetric Gauge Theories}
\label{ch:10}
We shall now move towards the Standard Model by looking at imposing supersymmetry to gauge theories. We shall first attempt to do so for a $U(1)$ gauge theory before moving on to generalise for an $SU(N)$ gauge theory. At the end, we will attempt to make QED supersymmetric.

\section{$U(1)$ gauge theory}
\label{ch:10:u1 gauge theory} 
For this theory, a vector multiplet is sufficient to illustrate the effects. We will use the vector multiplet that we have built earlier in Chapter \ref{ch:7} Section \ref{ch:7:applying onto supercharges}, $h = \{\pm 1/2, \pm 1\}$. Let us consider a photon field given by $A_\mu$ and its superpartner $\lambda$. The free Lagrangian for this theory is:
\begin{equation}
    \mathcal{L} = - \frac{1}{4} F_{\mu \nu}^{\mu \nu} + \lambda^\dagger i\Bar{\sigma}^\mu \partial_\mu \lambda
    \label{eqn:10:u1 gauge theory:free lagrangian without auxiliary field}
\end{equation}

Since the field strength $F_{\mu \nu}$ is neutral in charge, so is $\lambda$. Because of that, there is no need for us to replace $\partial_\mu$ with a covariant derivative as the free Lagrangian is gauge invariant on its own.

Let us determine the supersymmetric transformation rule of the fields. Since we want this to be a supersymmetric, we want $A_\mu$ to transform into $\lambda$ and back. Our first ansatz is 
\begin{equation}
    \delta A^\mu = \xi^\dagger \Bar{\sigma}^\mu \lambda + \lambda^\dagger \Bar{\sigma}^\mu \xi
    \label{eqn:10:u1 gauge theory:photon field transformation rule ansatz}
\end{equation}
where by dimensional analysis, $[\xi] = -1/2$. 

For the transformation rule of the photino, we need to ensure gauge invariance so we shall make use of the gauge invariant $F_{\mu\nu}$. Our ansatz is
\begin{equation}
    \delta \lambda = C F_{\mu \nu} \sigma^\mu \Bar{\sigma}^\nu \xi
    \label{eqn:10:u1 gauge theory:photino transformation rule ansatz}
\end{equation}
where the know that $\sigma^\mu \Bar{\sigma}^\nu \xi$ is a proper rank 2 tensor. As $\lambda$ is a left chiral (by choice), $\xi$ also has to be left chiral.

To determine the value of the constant $C$, we impose the invariance of the Lagrangian:
\begin{equation}
    \delta \left( -\frac{1}{4} F_{\mu\nu}F^{\mu\nu} \right) = -F_{\mu\nu}( \xi^\dagger \Bar{\sigma}^\nu \partial^\mu \lambda + \partial^\mu\lambda^\dagger \Bar{\sigma}^\nu \xi)
    \label{eqn:10:u1 gauge theory:photon kinetic variance}
\end{equation}
\begin{equation}
    \delta(i\lambda^\dagger \Bar{\sigma}^\mu \partial_\mu \lambda) = i C` lambda^\dagger \Bar{\sigma}^\rho \partial_\rho F_{\mu\nu}\sigma^\mu \Bar{\sigma}^\nu \xi
    \label{eqn:10:u1 gauge theory:photino kinetic variance}
\end{equation}

Putting them together,
\begin{equation}
    \delta\mathcal{L} = -F_{\mu\nu} \xi^\dagger \Bar{\sigma}^\nu \partial^\mu \lambda (1-2iC^*)
\end{equation}
which means that we should pick $C = i/2$.

Now as we move on to include the auxiliary fields, we need to hold ourselves back from adding the same $F$ auxiliary fields that we had in the Wess-Zumino model. They are inherently different theories and so we need to treat them differently. For example, although $A_\mu$ also has 2 degrees of freedom on-shell, off-shell, it has 3. $\lambda$ has 2 degrees of freedom on-shell and 4 degrees of freedom off-shell. We see here that for the photon-photino multiplet, there is an absence of 1 degree of freedom. Therefore, our auxiliary field has to be of 1 degree of freedom, which means it has to be a real scalar field. The Lagrangian contribution is
\begin{equation}
    \mathcal{L}_{aux} = \frac{1}{2} D^2
\end{equation}
where we need to note that $D$ is inherently charge-less as well because of the total gauge invariance.

Since $[D] = [F]$, we shall assume that it supersymmetrically transforms like $F$.
\begin{equation}
    \delta D = \partial_\mu \lc\xi^\dagger (-i\Bar{\sigma}^\mu) \lambda + h.c.\rc
\end{equation}

Since we see that $D$ is real, gauge-invariant, and transforms as a total derivative, we can add a linear term to the Lagrangian and not have it affect the overall invariance. This will come in handy when spontaneously breaking supersymmetry.
\begin{equation}
    \mathcal{L}_{FI} = \varsigma D
\end{equation}

To complete the algebra, we have to fix the transformation rule of the photino as we did in the Wess-Zumino model.

In total, the free SUSY $U(1)$ gauge theory is
\begin{equation}
    \mathcal{L} = -\frac{1}{4}F_{\mu\nu}F^{\mu\nu} + i \lambda^\dagger \Bar{\sigma}^\mu \partial_\nu \lambda + \frac{1}{2} D^2 + \varsigma D
    \label{eqn:10:u1 gauge theory:free lagrangian}
\end{equation}
with the following supersymmetric theories:
\begin{eqnarray}
    \delta A^\mu &=& \xi^\dagger \Bar{\sigma}^\mu \lambda + \lambda^\dagger \Bar{\sigma}^\mu \xi \nonext 
    \delta \lambda &=& \frac{1}{2} i F_{\mu \nu} \sigma^\mu \Bar{\sigma}^\nu \xi + D \varsigma \label{eqn:10:u1 gauge theory:transformation rules}\\
    \delta D &=& -i \xi^\dagger \Bar{\sigma}^\mu \partial_\mu \lambda + i \partial_\mu \lambda^\dagger \Bar{\sigma}^\mu \xi \nonumber
\end{eqnarray}

\section{$SU(N)$ gauge theories}
We shall now extend the work for $SU(N)$ gauge theories. Here, we shall seed the boson field $A_\mu$ (not necessarily the photon field from before) with a charge $g$. Because of that, our $\partial_\mu$ is now replaced by a covariant derivative.
\begin{equation}
    \partial_\mu \rightarrow D_\mu = \partial_\mu + ig A_\mu^a T_F^a
\end{equation}
where $T_F^a$ is the fundamental representation of the gauge theory.

The $A$ field transforms in the fundamental representation as
\begin{equation}
    A_\mu \rightarrow A^\prime_\mu = UA_\mu U^\dagger + \frac{1}{g}(\partial_\mu U)U^\dagger
\end{equation}
and the covariant derivative transforms in the adjoint representation.
\begin{equation}
    D_\mu \psi \rightarrow D^\prime_\mu \psi^\prime = U D_\mu \psi
\end{equation}

To make the Dirac spinors gauge invariant, we use the covariant derivative instead of the normal partial derivative as well.

Since the field strength of the gauge field is given by the following expression:
\begin{equation}
    F_{\mu\nu} = \partial_\mu A_\nu - \partial_\nu A_\mu - ig [A_\mu, A_\nu]
\end{equation}
where $F_{\mu_\nu}$ is a matrix in the Lie algebra, we can find that $F_{\mu\nu}$ transforms in the fundamental representation.
\begin{equation}
    F_{\mu\nu} \rightarrow F_{\mu\nu}^\prime = U F_{\mu\nu} U^\dagger
\end{equation}

\section{QCD in Weyl spinors}
\label{ch:10:qcd in weyl spinors}
Using the information from the previous section, we can express QCD using Weyl spinors. The Lagrangian of a quark is given by
\begin{eqnarray}
    \mathcal{L}_{QCD} &=& \Bar{\Psi}_D (i\gamma^\mu D_\mu - m)\Psi_D \nonext
    &=& \Bar{\Psi}_D (i\gamma^\mu \partial_\mu - g\gamma^\mu A_\mu - m) \Psi_D
\end{eqnarray}
where being $SU(3)$, $A_\mu = \frac{1}{2} \lambda^a A_\mu^a$.

Properly changing expanding the equation, we get
\begin{eqnarray}
    \mathcal{L}_{QCD} &=& i \chi^\dagger_{\Bar{q}} \Bar{\sigma}^\mu \left[ \partial_\mu -\frac{1}{2} igA^a_\mu (\lambda^a)^*\right] \chi_{\Bar{q}} + i \chi^\dagger_q \Bar{\sigma}^\mu \left[ \partial_\mu +\frac{1}{2} igA^a_\mu (\lambda^a)^*\right] \chi_q \nonext
    && - m(\chi_q\cdot\chi_{\Bar{q}} + \Bar{\chi}_{\Bar{q}}\cdot\Bar{\chi}_q)
\end{eqnarray}
where we see that the covariant derivative is different for $q$ and $\Bar{q}$, which means that they transform very differently and under nonequivalent representations. This is because the they have opposite charges!

\section{Free Abelian Vector multiplet $\times$ Free chiral multiplet}
\label{ch:10:free combination}
Let us now try to combine what we have done for the free multiplets and introduce interactions to them. The chiral multiplet is the multiplet with $\chi$, $\phi$, and $F$. To couplet the multiplets together, we introduce a $U(1)$ charge to the chiral multiplet.
\begin{equation}
    X = \exp[i q \Lambda(x)] X
\end{equation}
where $X$ is any of the particles in the chiral multiplet.

The free Lagrangian is now becomes
\begin{equation}
    \mathcal{L} = (D_\mu \phi)^\dagger (D^\mu \phi) + i \chi^\dagger \Bar{\sigma}^\mu D_\mu \chi + F^\dagger F - \frac{1}{4} F_{\mu \nu} F^{\mu \nu} + i\lambda^\dagger \Bar{\sigma}^\mu \partial_\mu \lambda + \frac{1}{2}D^2 + \varsigma D
    \label{eqn:10:free abeleian x free chiral:new free lagrangian}
\end{equation}
where some interaction terms come when we expand the covariant derivative. These terms are
\begin{equation}
    \left(\mathcal{L}_{int}\right)_1 = iq \phi A^\mu \partial_\mu \phi^\dagger - iq \phi^\dagger A^\mu \partial_\mu \phi - q\chi^\dagger \Bar{\sigma}^\mu A_\mu \chi
\end{equation}

In finding new interaction terms, we need to ensure the 4 constraints of Chapter \ref{ch:8} Section \ref{ch:8:interactions to consider}. The need for gauge invariance greatly narrows the options and the only remaining terms are
\begin{equation}
    \left(\mathcal{L}_{int}\right)_2 = c_1 \left(\phi^\dagger \chi \cdot \lambda + h.c.\right) + c_2 \phi^\dagger \phi D
\end{equation}

Adding $\left(\mathcal{L}_{int}\right)_2$ to Equation \ref{eqn:10:free abeleian x free chiral:new free lagrangian}, we will have the general free abelian vector multiplet $\times$ free chiral multiplet Lagrangian. The transformation rules for each particle is that in their own theories in Equations \ref{eqn:6:supersymmetric field transformations} and \ref{eqn:10:u1 gauge theory:transformation rules}, except that we cannot have both of them use the same infinitesimal parameter $\xi$. The vector multiplet will have an infinitesimal parameter $a\xi$, where $a$ is a constant. 

To determine the coefficients $a$, $c_1$, and $c_2$, we have to vary $(\mathcal{L}_{int})_1 + (\mathcal{L}_{int})_2$ and ensure that it either vanishes or gauges away as a total derivative, as what we did in Section \ref{ch:10:u1 gauge theory}. Going through the work, we will arrive at 
\begin{equation}
    a = -1 / \sqrt{2} \quad , \quad c_1 = -\sqrt{2} q \quad , \quad c_2 = -q
\end{equation}

However, now the auxiliary field does not close on its algebra. To fix this, we have to add some terms to the transformation rules of $F$ and $F^\dagger$.
\begin{equation}
    \delta F = -i \xi^\dagger \Bar{\sigma}^\mu D_\mu \chi + \sqrt{2}q \phi \Bar{\xi}\cdot\Bar{\lambda}
\end{equation}
where this means that the chiral auxiliary field transforms into spinors in the chiral and vector multiplets!

All in all, our Lagrangian of the free abelian vector multiplet $\times$ free chiral multiplet now reads
\begin{eqnarray}
    \mathcal{L} &=& D_\mu \phi^\dagger D^\dagger \phi + i \chi^\dagger \Bar{\sigma}^\mu D_\mu \chi + F^\dagger F - \frac{1}{4} F_{\mu \nu} F^{\mu \nu} + i \lambda^\dagger \Bar{\sigma}^\mu \partial_\mu \lambda \nonext
    && + \frac{1}{2} D^2 + D \varsigma - \sqrt{2} q (\phi^\dagger \chi \cdot \lambda + h.c.) - q\phi^\dagger \phi D
    \label{eqn:10:free abelian times free chiral lagrangian}
\end{eqnarray}

and the fields of this theory transforms supersymmetrically as
\begin{eqnarray}
    \delta A^\mu &=& -\frac{1}{\sqrt{2}} (\xi^\dagger \Bar{\sigma}^\mu \lambda + \lambda^\dagger \Bar{\sigma}^\mu \xi) \nonext
    \delta \lambda &=& -\frac{1}{2\sqrt{2}} F_{\mu\nu} \sigma^\mu \Bar{\sigma}^\nu \xi - \frac{1}{\sqrt{2}} D \xi \nonext
    \delta D &=& \frac{i}{\sqrt{2}} \xi^\dagger \Bar{\sigma}^\mu \partial_\mu \lambda - \frac{i}{\sqrt{2}} (\partial_\mu \lambda)^\dagger \Bar{\sigma}^\mu \xi \nonext
    \delta \phi &=& \xi \cdot \chi \nonext
    \delta \chi &=& -i (D_\mu \phi) \sigma^\mu (i\sigma^2) \xi^* + F \xi \nonext
    \delta F &=& -i \xi^\dagger \Bar{\sigma}^\mu D_\mu \chi + \sqrt{2} q\phi \Bar{\xi} \cdot \Bar{\lambda}
\end{eqnarray}

For there to be a superpotential for the single chiral multiplet, it is necessary to be holomorphic in $\phi$, which means that the $U(1)$ charge of the chiral multiplet is 0, $\implies \phi = \phi^\dagger$. However, if we have multiple chiral multiplets, we simply have to ensure that the combination of multiplets must have a vanishing overall gauge charge and we will have a superpotential that embeds all the information of their self interactions.

\section{Nonabelian free vector multiplet $\times$ free chiral multiplet}
\label{ch:10:nonableian free vector multiplet times free chiral multiplet}
To process is the same, where the only difference is that now we have to ensure that we keep track of the right indices for gauge fields and charges. After doing so, we will find that the interacting Lagrangian is
\begin{equation}
    \mathcal{L}_{int} = c_1\left([\phi^{\dagger^b} (T_F^a)^{bc} \chi^c] \cdot \lambda^a + h.c.\right) + c_2 [\phi^{\dagger^b} (T_F^a)^{bc} \phi^c] D^a
\end{equation}
where we will yet again find that
\begin{equation}
    c_1 = - \sqrt{2}g \quad , \quad c_2 = -g
\end{equation}

And with that, the Lagrangian becomes
\begin{eqnarray}
    \mathcal{L} &=& D_\mu \phi^\dagger D^\dagger \phi + i \chi^\dagger \Bar{\sigma}^\mu D_\mu \chi + F^\dagger F - \frac{1}{4} F_{\mu \nu} F^{\mu \nu} + i \lambda^\dagger \Bar{\sigma}^\mu \partial_\mu \lambda \nonext
    && + \frac{1}{2} D^2 - \sqrt{2} g (\phi^\dagger \chi \cdot \lambda + h.c.) - g\phi^\dagger \phi D
\end{eqnarray}
and just as with the abelian case, the transformation rule for $F$ has to be modified to
\begin{equation}
    \delta F = ... + \sqrt{2} g \phi^b (T_F^a)^{bc} \Bar{\xi} \cdot \Bar{\lambda}^a
\end{equation}



%\chapter*{Appendix}

\bibliographystyle{spphys}
\bibliography{reference.bib}
\end{document}
