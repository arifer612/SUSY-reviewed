%%%%%%%%%%%%%%%%%%%% author.tex %%%%%%%%%%%%%%%%%%%%%%%%%%%%%%%%%%%
%
% sample root file for your "contribution" to a contributed volume
%
% Use this file as a template for your own input.
%
%%%%%%%%%%%%%%%% Springer %%%%%%%%%%%%%%%%%%%%%%%%%%%%%%%%%%


% RECOMMENDED %%%%%%%%%%%%%%%%%%%%%%%%%%%%%%%%%%%%%%%%%%%%%%%%%%%
\documentclass[%
graybox,
vecphys,
]{svmult}

% choose options for [] as required from the list
% in the Reference Guide

\usepackage{type1cm}        % activate if the above 3 fonts are
                            % not available on your system
%
\usepackage{makeidx}         % allows index generation
\usepackage{graphicx}        % standard LaTeX graphics tool
                             % when including figure files
\usepackage{multicol}        % used for the two-column index
\usepackage[bottom]{footmisc}% places footnotes at page bottom


\usepackage{newtxtext}       % 
\usepackage{newtxmath}       % selects Times Roman as basic font

% see the list of further useful packages
% in the Reference Guide

\makeindex             % used for the subject index
                       % please use the style svind.ist with
                       % your makeindex program

%%%%%%%%%%%%%%%%%%%%%%%%%%%%%%%%%%%%%%%%%%%%%%%%%%%%%%%%%%%%%%%%%%%%%%%%%%%%%%%%%%%%%%%%%
% USER DEFINITIONS

%USER PACKAGES
\usepackage{slashed}        % Dirac slash notation
\usepackage{braket}         % Braket notation
\usepackage{bbold}

\newcommand{\nonext}{\nonumber \\}  % Used for eqnarrays
\newcommand{\td}{\text{d}}          % For differentials
\newcommand{\e}[1]{\text{e}^{#1}}   % Exponential
\newcommand{\lc}{\left(}            % Left bracket formatting
\newcommand{\rc}{\right)}           % Right bracket formatting
\newcommand{\iden}{\mathbb{1}}      % Identity
\newcommand{\col}[2]{\begin{pmatrix}{#1}\\{#2}\end{pmatrix}}


%%%%%%%%%%%%%%%%%%%%%%%%%%%%%%%%%%%%%%%%%%%%%%%%%%%%%%%%%%%%%%%%%%%%%%%%%%%%%%%%%%%%%%%%%

\begin{document}

\title*{Review of Supersymmetry}
% Use \titlerunning{Short Title} for an abbreviated version of
% your contribution title if the original one is too long
\author{Arif Er}
% Use \authorrunning{Short Title} for an abbreviated version of
% your contribution title if the original one is too long
\institute{Arif Er \at NUS, \email{e0032348@u.nus.edu}}
%
% Use the package "url.sty" to avoid
% problems with special characters
% used in your e-mail or web address
%
\maketitle

\abstract{A review and revision of my study of Supersymmetry I studied in the summer of 2020. This review will cover the topics covered in \textit{Supersymmetry DeMYSTiFied} by Labelle \cite{labelle2010supersymmetry}}

\setcounter{chapter}{1}
\chapter{Introduction to Weyl spinors}
\label{ch:2}

We begin the revision by recapping on the physics of the Weyl spinors and how they are relevant in our study of the Quantum Field Theories. We will then attempt to formulate possible Lorentz invariants from the Dirac spinors so that they can be used in the Lagrangian formalism. Lastly, we will look at the Van der Waarden notation, a more compact and useful notation for Weyl spinors especially in the context of Supersymmetry.

\section{The Dirac equation}
\label{ch:1:dirac equation}
Our starting point is the Dirac equation. It relates shows how one can obtain the eigenvalue of the momentum operator of a quantum particle.
\begin{equation}
    \gamma^\mu P_\mu \psi = m \psi \quad , P_\mu \equiv i \partial_\mu
    \label{eqn:2:dirac equation}
\end{equation}

Using the Dirac slash, it is identically
\begin{equation}
    \slashed P \psi = m \psi
    \label{eqn:2:slashed dirac equation}
\end{equation}

The Lagrangian for a Dirac particle is thus
\begin{equation}
    \mathcal{L}_{Dirac} = \Bar{\psi} (\gamma^\mu P_\mu - m) \psi
    \label{eqn:2:dirac lagrangian}
\end{equation}

The $\gamma^\mu$ used above are the Dirac matrices, $4\times4$ matrices that are built off the Pauli matrices.
\begin{equation}
    \gamma^0 =
    \begin{pmatrix}
        0 & \mathbb{1}\\
        \mathbb{1} & 0
    \end{pmatrix}
    \quad , \quad
    \gamma^i =
    \begin{pmatrix}
        0 & - \sigma^i\\
        \sigma^i & 0
    \end{pmatrix}
    \label{eqn:2:dirac matrices}
\end{equation}

Using the mostly negative signature metric (i.e. $\eta_\mu\nu = (+, -, -, -)$), the Dirac matrices are:
\begin{equation}
    \gamma^\mu = (\gamma^0, \gamma^i) \quad , \quad\gamma_\mu = (\gamma^0, -\gamma^i)
    \label{eqn:2:dirac matrices lowered}
\end{equation}

We were also introduced another new Dirac matrix, for the fact that it simplifies a large deal of work in the later part of our journey.
\begin{equation}
    \gamma_5 =
    \begin{pmatrix}
        \mathbb{1} & 0\\
        0 & \mathbb{1}
        \label{eqn:2:dirac matrix 5}
    \end{pmatrix}
\end{equation}

From the properties of Pauli matrices, we see some interesting results that would turn out to be central to the formulation of the framework.
\begin{eqnarray}
    \sigma^2 (\sigma^i)^T &=& - (\sigma^i) \sigma^2 \nonext
    \sigma^2 (\sigma^i)^* &=& - (\sigma^i) \sigma^2
    \label{eqn:2:pauli matrix property 1}
\end{eqnarray}
\begin{eqnarray}
    \sigma^2 (\sigma^i) \sigma^2 = - (\sigma^i)^T = - (\sigma^i)^*\nonext
    \therefore \sigma^2 (\sigma^i)^T \sigma^2 = - (\sigma^i)
    \label{eqn:2:sigma T and *}
\end{eqnarray}

We thus have the following representation of a vector weighted matrix
\begin{eqnarray}
    \Vec{A} \cdot \Vec{\sigma} \sigma^j &=& A^i \sigma^i \sigma^j \nonext
    &=& A^i (\sigma^j \sigma^i - [\sigma^i, \sigma^j]) \nonext
    &=& A^i \sigma^j \sigma^i - 2 i \varepsilon^{ijk}A^i \sigma^k \nonext
    &=& \sigma^j \Vec{A} \cdot \Vec{\sigma} - 2 i (\Vec{A} \times \Vec{\sigma})
    \label{eqn:2:vector weighted matrix}
\end{eqnarray}

\section{Dirac spinors}
These are reducible 4 component spinors. Their lowest representation is a 2 component spinor, a `left-chiral' and a `right-chiral' spinor. 
\begin{equation}
    \psi = \begin{pmatrix} \eta \\ \chi \end{pmatrix}
    \label{eqn:2:dirac spinor}
\end{equation}

Extracting the irreducible spinors is simple with a projection operator. Clearly, if
\begin{equation*}
    P_R \psi = \begin{pmatrix} \eta \\ 0 \end{pmatrix} 
    \quad , \quad
    P_L \psi = \begin{pmatrix} 0 \\ \chi \end{pmatrix}
\end{equation*}
The projection operators have to be:
\begin{equation*}
    P_R = \begin{pmatrix} \mathbb{1} & 0 \\ 0 & 0 \end{pmatrix}
    \quad , \quad
    P_L = \begin{pmatrix} 0 & 0 \\ 0 & \mathbb{1} \end{pmatrix}
\end{equation*}
which relates back to the $\gamma_5$ matrix in Eqn \ref{eqn:2:dirac matrix 5} as
\begin{equation}
    \gamma_5 = P_R - P_L
\end{equation}

We will use this to expand the Dirac equation into its irreducibles.
\begin{eqnarray}
    \gamma^\mu P_\mu \psi = m \psi
    &\implies&
    \begin{cases}
        (E \mathbb{1} + \Vec{\sigma} \cdot \Vec{p}) \chi = m \eta\\
        (E \mathbb{1} - \Vec{\sigma} \cdot \Vec{p}) \eta = m \chi
    \end{cases}
    \label{eqn:2:dirac equation in weyl spinors}
    \\
    &=&
    \begin{cases}
        \Bar{\sigma}^\mu P_\mu \chi = m \eta \\
        \sigma^\mu P_\mu \eta = m \chi
    \end{cases}
    \label{eqn:2:dirac equation in weyl spinors compact}
\end{eqnarray}
where we are introduced to $\sigma^\mu = (\mathbb{1}, \sigma^i)$ and $\Bar{\sigma}^\mu = (\mathbb{1}, - \sigma^i)$ to get to Equation \ref{eqn:2:dirac equation in weyl spinors compact} from Equation \ref{eqn:2:dirac equation in weyl spinors}

The Dirac Lagrangian in its irreducible forms is thus
\begin{equation}
    \mathcal{L}_{Dirac} = \chi^\dagger i \Bar{\sigma}^\mu \partial_\mu \chi + \eta^\dagger i \sigma^\mu \partial_\mu \eta - m(\chi^\dagger \eta + \eta^\dagger \chi)
    \label{eqn:2:dirac lagrangian in weyl spinors}
\end{equation}

This reveals the coupled nature of the Dirac spinor irreducibles, or at least if they are massive. The special case of the massless Dirac spinor are known Weyl spinors. Being massless, we get the Weyl equations that tell us the eigenvalues of the spinors with respect to the $\Vec{\sigma}\cdot\Vec{p}$ operator:
\begin{equation}
    \begin{cases}
        E \eta = \Vec{\sigma} \cdot \Vec{p} \eta\\
        E \chi = - \Vec{\sigma} \cdot \Vec{p} \chi
    \end{cases}
    \implies
    \begin{cases}
        \frac{\Vec{\sigma}\cdot\Vec{p}}{\vert p \vert} \eta = \eta\\
        \frac{\Vec{\sigma}\cdot\Vec{p}}{\vert p \vert} \chi = \chi\\
    \end{cases}
    \label{eqn:2:weyl equations}
\end{equation}

And very conveniently, making use of the fact that $\Vec{S}\cdot\hat{p}$ is the helicity operator, we see why $\eta$ and $\chi$ are called chiral spinors!
\begin{equation}
    \Vec{S} \cdot \hat{p} \eta = \frac{\hbar}{2} \eta
    \quad , \quad 
    \Vec{S} \cdot \hat{p} \chi = - \frac{\hbar}{2} \chi
    \label{eqn:2:helical spinors}
\end{equation}

\section{Lorentz invariances}
\label{ch:2:lorentz invariances}
We want to build Lorentz invariants made of Weyl spinors, so that we might add them as interactions when building Lagrangians of any theory that might come along. For this, we need to know how they transform. From the fact that the Dirac Lagrangian is necessarily Lorentz invariant, the terms of Equation \ref{eqn:2:dirac lagrangian in weyl spinors} also has to be. Let us look into each term carefully.

\subsection{Pure terms}
\label{ch:2:lorentz invariances:pure terms}
The `pure' terms are $\chi^\dagger\eta$ and $\eta^\dagger\chi$. We know how each spinor transforms under the Lorentz group:
\begin{eqnarray}
    \eta &\rightarrow& \lc\mathbb{1} + \frac{1}{2} i \Vec{\varepsilon}\cdot\Vec{\sigma} - \frac{1}{2} \Vec{\beta}\cdot\Vec{\sigma}\rc \eta
    \label{eqn:2:right chiral transformation} \\
    \chi &\rightarrow& \lc\mathbb{1} + \frac{1}{2} i \Vec{\varepsilon}\cdot\Vec{\sigma} + \frac{1}{2} \Vec{\beta}\cdot\Vec{\sigma}\rc \chi 
    \label{eqn:2:left chiral transformation}
\end{eqnarray}
The difference between the transformation of the left and right chiral spinors is very subtle. But this difference allows us to understand why the `pure' terms are invariant. Moving on to the remainder of the terms,

\subsection{Mixed terms}
\label{ch:2:lorentz invariances:mixed terms}
These are the terms with $\sigma^\mu$ in them: $\chi^\dagger i \Bar{\sigma}^\mu \partial_\mu \chi$ and $\eta^\dagger i \sigma^\mu \partial_\mu \eta$. We know from earlier that $\chi^\dagger\eta$ forms an invariant. Therefore the term that couples with $\chi^\dagger$ has to transform like a right chiral spinor. This means that $i\Bar{\sigma^\mu}\partial_\mu\chi$ transforms like a right chiral spinor. This is trivial to prove. One way is to recall the coupled irreducibles in Equation \ref{eqn:2:dirac equation in weyl spinors compact}. $m$ is an invariant, so the remaining terms have to transform in the same manner as each other. The other way is to explicitly find the transformation rule of $i \Bar{\sigma}^\mu \partial_\mu \chi$ to find out that it does indeed transform as a right chiral spinor. It is the same as $i \sigma^\mu \partial_\mu \eta$, which transforms neatly as a left chiral spinor.

Moreover, if we were to do an integration by parts (IbP) on these mixed terms, we will find that:
\begin{equation}
    - \partial_\mu(\chi^\dagger i \Bar{\sigma}^\mu) \chi = (i \Bar{\sigma}^\mu \partial_\mu \chi)^\dagger \chi
    \label{eqn:2:invariant ibp}
\end{equation}
is also in invariant! Clearly for Equation \ref{eqn:2:invariant ibp} to make sense with the results in the Section \ref{ch:2:lorentz invariances:pure terms} $i \Bar{\sigma}^\mu \partial_\mu \chi$ has to transform as a right chiral, which again agrees with what we did earlier.

\subsection{Using only left chirals}
\label{ch:2:lorentz invariances:using sigma2}
Using the properties of $\sigma^2$, we can create Lorentz invariants using only left chirals, without the need of any vector matrices. The transformation of ${\chi^\dagger}^T$ in the contraction $\chi^\dagger\eta$ is
\begin{eqnarray}
    {\chi^\dagger}^T &\rightarrow& \lc \iden + \frac{1}{2} i \Vec{\varepsilon} \cdot \Vec{\sigma} + \frac{1}{2} \Vec{\beta} \cdot \Vec{\sigma} \rc^* {\chi^\dagger}^T \nonext
    &=& \lc \iden - \frac{1}{2} i \Vec{\varepsilon} \cdot \Vec{\sigma} + \frac{1}{2} \Vec{\beta} \cdot \Vec{\sigma} \rc {\chi^\dagger}^T
    \label{eqn:2:left chiral dagger transformation}
\end{eqnarray}

To pair this with the left chiral $\chi$, we need to get it to transform as a right chiral, for which is Equation \ref{eqn:2:left chiral dagger transformation} but with the opposite sign. Multiplying throughout by $i \sigma^2$ does exactly that, as we did in Equation \ref{eqn:2:pauli matrix property 1}. Therefore, $i \sigma^2 {\chi^\dagger}^T$ transforms like a right chiral!

The invariant we will get out of this combination is done by taking the hermitian conjugate of it and pairing it with $\chi$.
\begin{equation}
    \lc i \sigma^2 {\chi^\dagger}^T \rc^\dagger = \chi^T \lc -i \sigma^2 \rc \chi
    \label{eqn:2:left chirals invariant}
\end{equation}

\subsection{Method}
\label{ch:2:lorentz invariances:method}
From what we have seen so far, Lorentz invariants made of Weyl spinors all require a combination of one left and right chiral spinor, with either of them being a hermitian conjugate. We can extend this further by making use of the fact that $\chi^\dagger \Bar{\sigma}^\mu \chi$ is a (1,0) tensor, since $P_\mu$ necessarily transforms as a vector under the Lorentz group.

\begin{question}{Creating a rank 2 covariant tensor from left chirals, without derivatives}
    This is a rather trivial exercise if we work out how the rank 1 covariant tensor $\chi^\dagger \Bar{\sigma}^\mu \chi$ transforms. Let us define the transformation of $\chi$ as $\chi \rightarrow A^{-1} \chi $, where A is naturally a transformation matrix.
    
    \begin{eqnarray*}
        \chi^\dagger \Bar{\sigma}^\mu \chi \rightarrow {\chi'}^\dagger \Bar{\sigma'}^\mu \chi' &=& \chi^\dagger {A^{-1}}^\dagger \Bar{\sigma}^\mu A^{-1} \chi
        \nonext
        &=& \chi^\dagger \Lambda^\mu_\nu \Bar{\sigma}^\nu\chi
    \end{eqnarray*}
    
    From this, we know that the transformation of $\Bar{\sigma}^\mu$ is
    \begin{equation*}
        \Bar{\sigma}^\mu \rightarrow A^\dagger \Lambda^\mu_\nu \Bar{\sigma}^\nu A
    \end{equation*}
    Likewise for $\sigma^\mu$, since $\eta^\dagger \sigma^\mu \eta$ is a rank 1 covariant tensor and $\eta^\dagger\chi$ is invariant (i.e. $\eta \rightarrow A^\dagger \eta$
    \begin{equation*}
        \sigma^\mu \rightarrow A^{-1} \Lambda^\mu_\nu \sigma^\nu {A^{-1}}^\dagger
    \end{equation*}
    
    Putting them together, we have
    \begin{equation*}
        \sigma^\mu \Bar{\sigma}^\nu \rightarrow A^{-1}\Lambda^\mu_\alpha\Lambda^\nu_\beta \sigma^\alpha \sigma^\beta A
    \end{equation*}
    
    Fitting this with a something that transforms as $? \rightarrow A^\dagger$ (i.e. a right chiral spinor) on the left and $? \rightarrow A^{-1}$ (i.e. a left chiral spinor) on the right, we will get a rank 2 covariant tensor! Since we are looking to populate these positions with only left chirals, the options are obvious.
    \begin{equation*}
        \chi^T (-i \sigma^2) \sigma^\mu \Bar{\sigma}^\nu \chi \rightarrow \Lambda^\mu_\alpha \Lambda^\nu_\beta \chi^T \lc -i \sigma^2\rc \sigma^\alpha \Bar{\sigma}^\beta \chi
    \end{equation*}
\end{question}

\section{Van der Waerden notation}
\label{ch:2:van der waerden notation}
Instead of the cumbersome $\pm i \sigma^2$ in between the chiral spinors, the Van der Waerden notation defines a new kind of dot product between chiral spinors. We saw how $\chi^T (-i \sigma^2) \chi$ and $\eta^\dagger (i \sigma^2) {\eta^\dagger}^T$ are invariants, so we shall define 2 dot products as such:
\begin{eqnarray}
    \chi \cdot \chi &\equiv& \chi^T (-i \sigma^2) \chi \nonext
    \Bar{\chi} \cdot \Bar{\chi} &\equiv& \chi^\dagger (i \sigma^2) {\chi^\dagger}^T \nonumber
\end{eqnarray}

Expanding the components,
\begin{eqnarray}
    \chi \cdot\chi = \chi_2 \chi_1 - \chi_1\chi_2 \nonext
    \Bar{\chi} \cdot \Bar{\chi} = \chi^\dagger_1 \chi^\dagger_2 - \chi^\dagger_2 \chi^\dagger_1
    \label{eqn:2:van der waerden expanded}
\end{eqnarray}

The good thing about this notation is that it shows that the dot product is now hermitian! So we can employ this directly in the Lagrangian fully knowing that we need not worry about any real-ness violations.
\chapter{A new notation}
\label{ch:3}
In this chapter, we will introduce a new notation for Weyl spinors that will be very useful in reading off the Lagrangian for interpretation and when working in the superfield approach.

\section{Indices}
\label{ch:3:indices}
We will begin off by defining the index notation. As we will mostly be working in the left-chiral basis, it will be the basis of which our notation will be built off from. We will begin from the invariant $\eta^\dagger\chi$. We will define as a contraction of
\begin{equation}
    \eta^\dagger \chi \equiv \eta^\dagger \chi_a
    \label{eqn:3:first definition}
\end{equation}

The hermitian conjugate of the right chiral spinor is defined as having an upper un-dotted index. We will define the right chiral spinor as having an upper dotted index.
\begin{equation}
    \eta \equiv \Bar{\eta}^{\Dot{a}} \implies \eta^a \equiv \lc \Bar{\eta}^{\Dot{a}}\rc ^\dagger
\end{equation}

With this, we can generalise the rest and see that the other fundamental invariant is
\begin{equation}
    \chi^\dagger\eta \equiv \Bar{\chi}_{\Dot{a}}\Bar{\eta}^{\Dot{a}}
\end{equation}

We will adopt a notation convention, that contractions between un-dotted indices are carried from top down, and for dotted indices from bottom up, as we see in the two definitions above.

\section{Raising and lowering indices}
We will make use of the fact that $(-i\sigma^2)_{ba}(i\sigma^2)^{ab} = \mathbb{1}$ allows us to define a metric to raise and lower the indices. To raise the indices,
\begin{equation}
    \Bar{\chi}^{\Dot{a}} \equiv (i\sigma^2)^{\Dot{a}b}\chi_b^\dagger = (i\sigma^2)^{\dot{a}\dot{b}}\Bar{\chi}_{\dot{b}}
    \label{eqn:3:raising operation}
\end{equation}
With this, we can see that explicitly,
\begin{equation}
    \Bar{\chi}^{\dot{1}} = \Bar{\chi}_{\dot{2}} = \chi_2^\dagger \quad , \quad \Bar{\chi}^{\dot{2}} = - \Bar{\chi}_{\dot{1}} = -\chi_1^\dagger
    \label{eqn:3:explicit raising operation}
\end{equation}

Likewise, lowering the indices is just as similar:
\begin{equation}
    \chi_b = (-i\sigma^2)_{ba}\chi^a
    \label{eqn:3:lowering operation}
\end{equation}

With this, the Van der Waerden dot product is
\begin{equation}
    \eta \cdot \chi = \eta^1 \chi_1 + \eta^2 \chi_2 = \eta_2\chi_1 - \eta_1\chi_2
\end{equation}

\section{The epsilon metric}
\label{ch:3:the epsilon metric}
To clean up the $(i\sigma^2)$ that is plaguing our notation, let us define
\begin{eqnarray}
    \varepsilon^{ab} &\equiv& (i\sigma^2)^{ab} \nonext
    \varepsilon^{\dot{a}\dot{b}} &\equiv& (i\sigma^2)^{\dot{a}\dot{b}} \nonext
    \varepsilon_{ab} &\equiv& (-i\sigma^2)_{ab} \nonext
    \varepsilon_{\dot{a}\dot{b}} &\equiv& (-i\sigma)^2_{\dot{a}\dot{b}} 
\end{eqnarray}

As $i\sigma^2$ is completely anti-symmetric, so is $\varepsilon$. Moreover, the contraction of $\varepsilon$ with itself is naturally
\begin{equation}
    \varepsilon^{ab}\varepsilon_{bc} = -\varepsilon^{ba}\varepsilon_{bc} = - \varepsilon^{ab}\varepsilon_{cb} = \varepsilon^{ba}\varepsilon_{cb} = \delta^a_c
\end{equation}

\section{$\sigma^\mu$ and $\Bar{\sigma}^\mu$ indices}
\label{ch:3:sigma indices}
We know that $i\sigma^\mu \eta$ is a left chiral spinor, so $i \sigma^\mu$ has to lower a dotted index to an un-dotted one. Likewise, $i\Bar{\sigma}^\mu \chi$ is a right chiral spinor, so $i \Bar{\sigma}^\mu$ raises an un-dotted index to a dotted one. We have:
\begin{equation}
    \Bar{\sigma}^\mu \equiv \left(\Bar{\sigma}^\mu\right)^{\dot{a}b} \quad , \quad \sigma^\mu \equiv \left(\sigma^\mu\right)_{a\dot{b}}
\end{equation}

We can see obtain this through $\varepsilon$
\begin{eqnarray}
    (i \sigma^2) \sigma^\mu (i \sigma^2) &=& \varepsilon^{ca}(\sigma^\mu)_{a\dot{b}}\varepsilon^{\dot{b}\dot{d}} \nonext
    &=& (\Bar{\sigma}^\mu)^{c\dot{d}} \nonext
    &=& - (\bar{\sigma}^\mu)^{\dot{d}c} \nonext
    &=& - \left(\bar{\sigma}^\mu\right)^T
\end{eqnarray}

With this, we can that supersymmetric invariants with $\sigma^\mu$ or $\bar{\sigma}^\mu$ needs to have spinors with both dotted and un-dotted indices, of the same generation:
\begin{equation}
    \bar{\chi}_{\dot{a}} \left(\Bar{\sigma}^\mu\right)^{\dot{a}b} \lambda_b \quad , \quad \chi^a (\sigma^\mu)_{a\dot{b}} \bar{\lambda}^{\dot{b}}
\end{equation}
\chapter{Weyl, Majorana, and Dirac spinors}
\label{ch:4}

In this chapter, we look at close relations between Weyl, Majorana, and Dirac spinors and how we can jump between one to the other (and when we should not). The advantages of understanding this is that it helps paint a clearer picture behind the interpretation of these rather `abstract' representations of what particles are. 

\section{Particle-antiparticle}
\label{ch:4:particle-antiparticle}
We will begin this chapter by looking at the intricacies of particle-antiparticle existence. The source of their coexistence is by the natural imposition of charge-parity-time-reversal (CPT) invariance on the theory of particles. The particle-antiparticle pair would ensure that total charge, parity, and time-reversal is upheld. In the following text, the particle will be denoted by the subscript $p$ whereas the antiparticle will be denoted by the subscript ${\Bar{p}}$. Their representations as Dirac spinors (in terms of the irreducible Weyl spinors) are
\begin{equation*}
    \psi_p = \col{\eta_p}{\chi_p} \quad , \quad \psi_{\Bar{p}} = \col{\eta_{\Bar{p}}}{\chi_{\Bar{p}}}
\end{equation*}

It should be necessary to mention that it is not the case that $\eta_p = \eta_{\Bar{p}}$ and $\chi_p=\chi_{\Bar{p}}$.

The relation between the conjugate Dirac spinor and its barred transpose used in the Lagrangian is
\begin{equation}
    \psi_{\Bar{p}} = \psi_p^C = C {\Bar{\psi}_p}^T \quad , \quad 
    C = - i \gamma^2 \gamma^0 
    = \begin{pmatrix} i \sigma^2 & 0 \\0 & - i\sigma^2\end{pmatrix}
    \label{eqn:4:particle-antiparticle:conjugate relations}
    \nonumber
\end{equation}

With this, 
\begin{equation}
    \psi_p^C = \col{i \sigma^2 {\chi_p^\dagger}^T}{-i\sigma^2{\eta_p^\dagger}^T}
    \implies 
    \begin{cases}
        \eta_{\Bar{p}} = i \sigma^2 {\chi_p^\dagger}^T \\
        \chi_{\Bar{p}} = - i \sigma^2 {\eta_p^\dagger}^T
    \end{cases}
    \label{eqn:4:particle-antiparticle:particle-antiparticle relations}
\end{equation}

We now see how intricately related the Weyl spinors of the particle and antiparticle are. One very important thing to point out is that from Equation \ref{eqn:4:particle-antiparticle:particle-antiparticle relations}, we can see that our discussion in Sec \ref{ch:2:lorentz invariances:using sigma2} agrees that $i \sigma^2 {\chi_p^\dagger}^T$ behaves as a right chiral and $- i \sigma^2 {\eta_p^\dagger}^T$ as a left chiral! Using this knowledge, we can get rid of any explicit right chiral representations in the Dirac spinor and simply express it as
\begin{equation}
    \psi = \col{i \sigma^2 {\chi_{\Bar{p}}^\dagger}^T}{\chi_p}
    \label{eqn:4:particle-antiparticle:dirac spinor as left chirals}
    \nonumber
\end{equation}

As mentioned at the start, the particle-antiparticle relations only exists because of the CPT invariance imposed on the Lagrangian. The other necessary constraint is that the Lagrangian needs to be real. (i.e. $\mathcal{L^\dagger} = \mathcal{L}$) This constraint tells us that the Lagrangian should either have both the hermitian conjugates of any chiral spinors, or none at all. The interpretation of this in QFT is very physical. There has to either have both the annihilation and creation operator of a particle, or none at all. Number operators will thus be a conserved operation. 

The simplest contributor to the Lagrangian that has both left chiral spinors and ensures CPT and reality invariance is
\begin{equation*}
    \mathcal{L} = \chi^\dagger i \sigma^\mu \partial_\mu \chi
\end{equation*}
There is both a left chiral particle spinor creation and annihilation field operator in this Lagrangian.

However, using the relation in Equation \ref{eqn:4:particle-antiparticle:particle-antiparticle relations}, the same Lagrangian then becomes
\begin{equation*}
    \mathcal{L} = \eta_{\Bar{p}}^T (i \sigma^2) i \sigma^\mu \partial_\mu \chi
\end{equation*}
which is a left chiral particle spinor and right chiral antiparticle spinor creation field operator! A very thought-provoking interpretation of the particle-antiparticle relationship.

Insofar as we have used the term Weyl spinor, we have used it to identify particles that are both eigenstates of the helicity operator and the chiralty operator. However, there is a very subtle difference between the two that paints very different pictures of what a Weyl spinor really is. As eigenstates of the helicity operator, Weyl spinors are necessarily massless as shown in \ref{eqn:2:helical spinors}. However, as eigenstates of the chiralty operator, they are simply eigenstates with fixed transformation rules under the $SU(2) \times SU(2)$ Lorentz group as in Equations \ref{eqn:2:right chiral transformation} and \ref{eqn:2:left chiral transformation}. Thus, there is no constraint on them being massless. They can be as massive as they need be, as long as they are eigenstates that of the Lorentz group. However, for the sake of continuing the discussion regarding massive particles using the Weyl spinor representation, we shall adopt the convention of the latter, while duly keeping in mind that actual Weyl spinors are necessarily massless.

Returning to the CPT invariance, we now see that we have a scheme that relates $\eta_p$ to $\chi_{\Bar{p}}$ and its conjugates. Through the Lorentz transformation, it is also possible (\textbf{for massive particles}) for the chiralty of the particle to change, i.e from $\eta_p$ to $\chi_p$ and vice versa. These 4 particles are thus related to each other as a multiplet that must exist as a collective state. It is because of this fact that we are allowed to express the right chiral particle as the left chiral antiparticle with impunity. This is evident in how the mass term of the Lagrangian can be expressed in either of the following representations:
\begin{eqnarray}
    m \Bar{\psi}\psi
    &=& m (\chi^\dagger \eta + \eta^\dagger \chi) \nonext
    &=& m (\chi\cdot\chi + \Bar{\chi}\cdot\Bar{\chi})
\end{eqnarray}

\section{Majorana spinors}
\label{ch:4:majorana}
The Majorana is a special subset of (massive) Dirac spinors. Its antiparticle state is the same as its particle state, i.e. $\eta_p = \eta_{\Bar{p}}$ and $\chi_p = \chi_{\Bar{\chi}}$. Unlike the general Dirac spinor, we now have 2 degrees of freedom instead of 4. The Majorana spinor in left chiral representation is
\begin{equation}
    \psi_M = \col{i \sigma^2 {\chi_p^\dagger}^T}{\chi_p}
\end{equation}

As good as the Majorana and Weyl representations are, it is not possible to build actual theories using them only as parity is not conserved. In the Lagrangian formalism of strictly Weyl or Majorana spinors, the mass terms will only be mass terms of left chirals, with no way of satisfying the parity between left and right chirals.

Looking at the two from another angle, we see that
\begin{equation*}
    \begin{cases}
        \Bar{\psi}_M \psi_M = \chi\cdot\chi + \Bar{\chi}\cdot\Bar{\chi} \\
        \Bar{\psi}_M \gamma_5 \psi_M = - \chi\cdot\chi + \Bar{\chi}\cdot\Bar{\chi}
    \end{cases}
\end{equation*}
which with some simple manipulation and generalisation, simply gives us
\begin{equation*}
\begin{cases}
    \lambda\cdot\chi = \Bar{\Lambda}_M P_L\psi_M\\
    \Bar{\lambda}\cdot\Bar{\chi} = \Bar{\Lambda}_M P_R \psi_M
\end{cases}
\end{equation*}

Lastly, making use of the fact that $\gamma^\mu$ can be represented off-diagonally as
\begin{equation}
    \gamma^\mu = \begin{pmatrix} 0 & \Bar{\sigma}^\mu \\ \sigma^\mu & 0 \end{pmatrix}
\end{equation}
we have

\begin{equation*}
    \begin{cases}
        \Bar{\psi}_M \gamma^\mu \Lambda_M = \chi^\dagger \Bar{\sigma}^\mu \lambda - \lambda^\dagger \Bar{\sigma}^\mu \chi \\
        \Bar{\psi}_M \gamma_5\gamma^\mu \Lambda_M = \chi^\dagger \Bar{\sigma}^\mu \lambda + \lambda^\dagger \Bar{\sigma}^\mu \chi
    \end{cases}
\end{equation*}
\begin{equation*}
    \implies
    \begin{cases}
        \chi^\dagger \Bar{\sigma}^\mu \lambda = \Bar{\psi}_M P_R \gamma^\mu \Lambda_M\\
        \lambda^\dagger \Bar{\sigma}^\mu \chi = - \Bar{\psi}_M P_L \gamma^\mu \Lambda_M
    \end{cases}
\end{equation*}
a very neat relation between the Weyl and Majorana representations of the \textbf{massive} Majorana spinor.
\chapter{Building the Lagrangian}
\label{ch:5}

We will build attempt at building the most basic Lagrangians with the invariants and constraints from the previous chapters. Several constraints on the Lagrangian will have to be imposed to ground our discussion in renormalisable theories.

\section{Dimensionfull Lagrangians}
In building the Lagrangian, we shall keep to renormalisable theories where $D = 4$ is maximally the further we will go in dimensions. The dimensions are defined in terms of powers of energy and as should be, the natural units are 1 (thus being dimensionless). With this, scalar fields are of dimension 1, derivatives are of dimension 1, fermion fields are of dimension $3/2$. Simple dimensional analysis will give us these.

\section{The simplest Lagrangian}
Let us consider the simplest toy model we can make -- a single free massless fermionic pair and a single free massless bosonic pair. They do not interact with each other (this will be introduced in Chapter \ref{ch:6}). The Lagrangian is simply
\begin{equation}
    \mathcal{L} = \partial_\mu \phi \partial^\mu \phi^\dagger + \chi^\dagger i \Bar{\sigma}^\mu \partial_\mu \chi
\end{equation}

The transformation of these fields are
\begin{eqnarray}
    \phi &\rightarrow& \phi + \delta \phi \nonext
    \chi &\rightarrow& \chi + \xi \chi \nonumber
\end{eqnarray}

Let us bring in the postulate of supersymmetry -- that bosons will transform into fermions and vice versa.
\begin{equation}
    \delta \phi \propto \xi \chi \quad , \quad \xi \ll 1
    \label{eqn:5:boson transformation approx}
\end{equation}

For Equation \ref{eqn:5:boson transformation approx} to satisfy the dimensionality of both sides of the equation, we see that $\xi$ has to be a Grassmann spinor of dimension $-1/2$. To actually determine the proportionality of the relationship in Equation \ref{eqn:5:boson transformation approx}, we have to impose the Lorentz invariance of the Lagrangian to obtain any more information.

Since $\xi$ is spinor, we have the freedom to pick a left chiral spinor, so we can have
\begin{equation}
    \delta \phi = \xi \cdot \chi
    \label{eqn:5:boson transformation}
\end{equation}
which is fully Lorentz invariant and a valid term in a Lagrangian.

We move on to the transformation of the fermion.
\begin{equation}
    \delta \chi = -i (\partial\phi) \sigma^\mu (i\sigma^2) \xi^*
    \label{eqn:5:fermion transformation}
\end{equation}
where we obtained this the same way, by imposing the equality of dimensions on both sides of the equation, Lorentz invariablity, and the reality of the Lagrangian.




%\chapter*{Appendix}

\bibliographystyle{spphys}
\bibliography{reference.bib}
\end{document}
