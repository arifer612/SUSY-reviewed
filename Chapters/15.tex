\chapter{The Minimal Supersymmetric Standard Model}
  \label{ch:15}

In this chapter, we will combine everything that we have done before and extend the sQED into a supersymmetric SM. For us to do so, we will have to first recall how the SM Lagrangian was built up before attempting to do so using the superspace formalism since we now know how much simpler it is to do so using it. Near the end, we will see that this would not be enough to fully build the Lagrangian mathematically -- we would need to weed out some unexpected invariants. However, as we will see after that, all of these may be neatly embedded in the addition of a new $U(1)$ global charge. 

\section{Review of the SM}
  \label{ch:15:review of SM}

The gauge group that builds the standard model is the $SU(3)_C \otimes SU(2)_L \otimes U(1)_\gamma$ group. The Higgs mechanism breaks the symmetry of $SU(3)_\gamma \otimes U(1)_\gamma$. In this theory, the fermions and scallar Higss transform in the fundamental representation. The gauge bosons transform in the adjoint fundamental representation. Roughly, the Lagrangian is read as
\begin{equation}
    \mathcal{L}_{SM} = T_{Fermions} + T_{GB} + T_{Higgs} + \Phi_{Higgs} + \Phi_{Yukawa} 
\end{equation}
where the Yukawa interactions refer to the coupling between the Higgs boson(s) and the fermions.

The Higgs mechanism is a complex doublet, which gives us 4 degrees of freedom. On spontaneous symmetry breaking, they become 3 massive particles and 1 massless particle. Fermion masses are generated by the spontaneous symmetry breaking of the Yukawa interactions. Neutrino masses may be added by modifying the Lagrangian slightly, but it does not affect the overall equations of motions so let us put it aside for our discussion. 

  \subsection{Building the SM Lagrangian}
  Before continuing, let us write down the quantum numbers of the fermions and Higgs bosons to figure out how their kinetic and interaction terms will look like. It is important to keep in mind that the fermions of the $SU(2)_L$ gauge are chiral so their quantum numbers are different for each chiralty.
    \subsubsection{$T_{Fermions}$}
    \label{ch:15:review of sm:t_fermions}

    \begin{equation}
      \Bar{\psi} i \gamma^\mu D_\mu \psi
    \end{equation}

    Let us consider the right-chiral electorn state $\eta_e$: It has quantum number $\ket{\mathbb{1}, \mathbb{1}, -2}$ in the $SU(3)_C \otimes SU(2)_L \otimes U(1)_\gamma$, i.e. a colour singlet, a weak group singlet, and a hypercharge of $-2$. A handy relation to note is that the hypercharge of a weak group singlet is twice that of the electric charge ($Y = 2Q$).

    To write the kinetic energy fo the right chiral electron, we have to apply the projection operator $P_R$ on $\psi_e$ and we will find that
    \begin{equation}
      \psi_{e_R} = \Bar{\psi}_{e_R} i \gamma^\mu D_\mu \psi_{e_R}
    \end{equation}
    where the covariant derivative here is
    \begin{equation}
      D_\mu = \partial_\mu + \frac{1}{2} i g^\prime Y B_\mu(x)
    \end{equation}

    Let us now consider the left-chiral fields $\psi_{e_L}$ and $\psi_{\nu_eL}$. They form a weak doublet so:
    \begin{equation}
      \psi_L \equiv \col{\psi_{\nu_eL}}{\psi_eL}
    \end{equation}
    where the order is defined such that the eigenvalues of $Q - \tau_3$ is the same for both particles. ($\tau_3$ = $\sigma_3$ but is labelled as such so as to differentiate the usage between Weyl spinors and doublet states).

    The hypercharge of an $SU(2)$ doublet is $Y = 2 (Q-\tau_3) = -1$. The quatum numbers are $\ket{\mathbb{1}, \mathbb{2}, -1}$. Thus,
    \begin{equation}
      D_\mu \psi_L = (\partial_\mu + \frac{1}{2} i g \tau^i W^i_\mu - \frac{1}{2} g^\prime B_\mu) \psi_L
    \end{equation}
    where $W_\mu^i$ are the $SU(2)$ gauge fields.

    Explicitly,
    \begin{equation}
      D_\mu \psi_L = \left[ \partial_\mu + \frac{1}{2} i g
        \begin{pmatrix}
          W_{3\mu} & W_{1\mu} - i W_{2\mu} \\
          W_{1\mu} + i W_{2\mu} & -W_{3\mu}
        \end{pmatrix} - \frac{1}{2} i g^\prime
        \begin{pmatrix}
          B_\mu & 0 \\ 0 & B_\mu
        \end{pmatrix}
        \right] \psi_L
      \label{eqn:15:review of sm:explicit d mu}
    \end{equation}

    Consider the up and down quarks:
    The left chiral components form an $SU(2)_L$ doublet but since they are quarks, they have a colour charges. The quantum number is $\ket{\mathbb{3}, \mathbb{2}, \frac{1}{3}}$. The doublet will be represented as
    \begin{equation}
      \psi_{Q_L} = \col{\psi_{u_L}}{\psi_{d_L}}
    \end{equation}

    Having an $SU(3)_C$ charge, its covariant derivative is
    \begin{equation}
      D_\mu = \partial_\mu + i g_S \lambda^a G^a_\mu + \frac{1}{2} i g \tau^i W_\mu^i + \frac{1}{6} i g^\prime B_\mu) + h.c.
    \end{equation}

    Consider: Right chiral up and down quarks.

    Unlike the left chiral up and down quarks, these are weak singlets, so we shall have to treat each of them individually.
    \begin{eqnarray}
      SU(3)_C \otimes SU(2)_L \otimes U(1)_\gamma \ket{\psi_{U_R}} &=& \ket{\mathbb{3}, \mathbb{1}, 4/3} \nonext
      SU(3)_C \otimes SU(2)_L \otimes U(1)_\gamma \ket{\psi_{U_L}} &=& \ket{\mathbb{3}, \mathbb{1}, -2/3} \nonumber
    \end{eqnarray}
    Their covariant derivatives are
    \begin{equation}
      D_\mu \psi_{u_R} = \left(\partial_\mu + \frac{1}{2} i g_S \lambda^a G_\mu^a + \frac{2}{3} i g^\prime B_\mu\right) \psi_{u_R}
    \end{equation}
    \begin{equation}
      D_\mu \psi_{d_R} = \left(\partial_\mu + \frac{1}{2} i g_S \lambda^a G_\mu^a - \frac{1}{3} i g^\prime B_\mu\right) \psi_{d_R}
    \end{equation}

    \subsubsection{$T_{GB}$}
    As we have done earlier, the gauge invariant kinetic energy of the gauge bosons are
    \begin{equation}
      T_{GB} = -\frac{1}{4} F^{\mu\nu}F_{\mu\nu} - \frac{1}{2} Tr\left(W^{\mu\nu}W_{\mu\nu}\right) - \frac{1}{2} Tr\left(G^{\mu\nu}G_{\mu\nu}\right)
    \end{equation}
    where the gauge fields are
    \begin{eqnarray}
      F_{\mu\nu} &=& \partial_\mu B_\nu - \partial_\nu B_\mu \nonext
      W_{\mu\nu} &=& \partial_\mu W_\nu - \partial_\nu W_\mu - ig [ W_\mu, W_\nu] \nonext
      G_{\mu\nu} &=& \partial_\mu G_\nu - \partial_\nu G_\mu - ig_s [G_\mu, G_\nu] \nonumber
    \end{eqnarray}
    and the traces are needed because $W_\mu$ and $G_\mu$ are matrix representations.

    \subsubsection{$T_{Higgs}$}
    The quantum numbers ofr the Higgs fields are $\ket{\mathbb{1}, \mathbb{2}, 1}$. Being a doublet, we will define it as
    \begin{equation}
      H \equiv \col{H^+}{H^0}
    \end{equation}

    The kinetic energy of this fields are simply
    \begin{equation}
      (D_\mu H)^\dagger (D^\mu H)
    \end{equation}
    where
    \begin{equation}
      D_\mu H = \left(\partial_\mu + \frac{1}{2} i g \tau^i W^i_\mu + \frac{1}{2} i g^\prime B_\mu \right) H
    \end{equation}

    \subsubsection{Yukawa terms}
    We need to keep in mind that the interaction terms we are planning to introduce need to observe gauge invariance, i.e. the total quantum numbers should read $\ket{\mathbb{1}, \mathbb{1}, 0}$.

    These new invariants are:
    \begin{equation}
      \Phi_{Yukawa} = y_e^{ij} (\bar{\psi}_L)_i H (\psi_{e_R})_j + y_d^{ij} (\bar{\psi}_{Q_L})_i H (\psi_{d_R})_j + y_u^{ij} (\bar{\psi}_{Q_L})_i H (\psi_{u_R})_j + h.c.
    \end{equation}

    \subsubsection{Higgs potential}
    Since the dimensions of the Higgs field is $1$, we can get the following invariants in the potential:
    \begin{equation}
      \Phi_H = -\mu^2 H^\dagger H + \lambda (H^\dagger H )^2
    \end{equation}
    where conventionally, $\mu^2$ and $\lambda > 0$.

  \subsection{Spontaneous symmetry breaking of the SM}
  \label{ch:15:review of sm:spontaneous breaking of sm}

  To break the symmetry, we need to recognise that fact that $\Phi_{Higgs}$ has a minimum when $\braket{V^\prime(H)} = 0$. This is only possible for positive $\mu^2$ and $\lambda$, which explains the convention. With this, we are then able to shift the Higgs field with respect to its minimum For which we shall define as $\braket{H^\dagger H}_{min} \equiv \nu^2$, where we get the minimum $H$ values by by the condition that the first derivative of $V$ vanish.
  \begin{equation}
    H_0^\prime \equiv H_0 - \nu \quad , \quad \implies \braket{H_0^\prime} = 0
  \end{equation}

  \begin{equation}
    \therefore D_\mu H = D_\mu \col{H^+}{H_0^\prime} + D_\mu \col{0}{\nu}
  \end{equation}

  Making use of the explicit form of $D_\mu$ in Equation \ref{eqn:15:review of sm:explicit d mu},
  \begin{equation}
    D_\mu \col{0}{\nu} = \frac{1}{2} i \nu \col{gW_{1\mu} - igW_{2\mu}}{-gW_{3\mu} + g^\prime B_\mu}
  \end{equation}

  In the Lagrangian,
  \begin{equation}
    (D_\mu H)^\dagger (D^\mu H) = \frac{1}{4} \nu^2 \left[ g^2 (W_{1\mu}^2 + W_{2\mu}^2 ) + (gW_{2\mu} - g^\prime B_\mu)^2\right]
  \end{equation}

  With this in the potential, we can define a new set of massive particles to assign these masses too.
  \begin{equation}
    (D_\mu H)^\dagger (D^\mu H) = m_W^2 W^{+\mu}W^-_\mu + \frac{1}{2} m_Z^2 Z_\mu Z^\mu
  \end{equation}
  which are what we now call the gauge bosons: the $W^\pm$ bosons defined as $W^\pm_\mu \equiv \frac{1}{\sqrt{2}}(W_{1\mu} \pm iW_{2\mu})$ with mass $\frac{1}{2} \nu^2 g^2$; and the $Z$ bosons defined as $Z_\mu \equiv \frac{1}{\sqrt{g^2 + g^{\prime 2}}} (gW_{3\mu} - g^\prime B_\mu )$ with mass $\frac{1}{2} \nu^2 (g^2 + g^{\prime 2})$.

  The last boson field is the photon field is defined as being orthogonal to the $Z$ field, $A_\mu \equiv \frac{1}{\sqrt{g^2 + g^{\prime 2}}} (g^\prime W_{3\mu} + g B_\mu)$ and is massless.

  Seeing that everything here can be expressed in terms of $g$ and $g^\prime$, and that the $Z_\mu$ and $A_\mu$ fields are dependent on the square-root of the their squared sum, it suggests that it may be neater to bring in some trigonometric relations.
  \begin{equation}
    \tan \theta_W \equiv \frac{g^\prime}{g}
  \end{equation}
  where $\theta_W$ is known as the Weinberg mixing angle.

  With that, the boson fields are
  \begin{eqnarray}
    z_\mu &=& \cos\theta_W W_{3\mu} - \sin\theta_W B_\mu \nonext
    A_\mu &=& \sin\theta_W W_{3\mu} + \cos\theta_W B_\mu \nonumber
  \end{eqnarray}

  We may invert the equation to get $B_\mu$ and $W_{3\mu}$ in terms of $Z_\mu$ and $A_\mu$. Since we know that the coupling constant of $A_\mu$ is $e$ and that the coupling constant of $B_\mu$ is $g^\prime$, this means that $e = g^\prime \cos\theta_W$. Likewise with $W_{3\mu}$ having a coupling constant of $g$, $e = g \sin\theta_W$ which is expected. It also gives us the relationship between $m_W$ and $m_Z$ to be $m_W = \cos \theta_W m_Z$.


\section{Building the MSSM}
\label{ch:15:building the MSSM}
Let us shift our focus back to building the MSSM. We shall work only in left chiral spinors, so that means we shall be exchanging the right-chiral electron field for the corresponding left chiral positron field. The hypercharge of this field will have an opposite sign since the electric charge of a positron is negative that of the electron. Note that changing from the right chiral to left chiral representation does not change the $SU(3)_C$ or $SU(2)_L$ quantum numbers, so the positron will still remain both a colour and weak group singlet.

  \subsection{The left chiral superfields of the MSSM}
  \label{ch:15:building the MSSM:left chiral superfields}

  The left chiral superfields we will be using are:
  \begin{equation}
    \mathcal{E}_1 = \tilde{\phi}_{\bar{e}} + \theta \cdot \chi_{\bar{e}} + \frac{1}{2} \theta\cdot\theta F_{\bar{e}}
  \end{equation}
  \begin{equation}
    \mathcal{L}_1 = \col{\tilde{\phi}_{\nu_e}}{\tilde{\phi}_e} + \theta \cdot \col{\chi_{\nu_e}}{\chi_e} + \frac{1}{2} \theta\cdot\theta \col{F_{\nu_e}}{F_e}
  \end{equation}
  \begin{equation}
    \mathcal{U}_1 = \tilde{\phi}_{\bar{u}} + \theta\cdot\chi_{\bar{u}} + \frac{1}{2} \theta\cdot\theta F_{\bar{u}}
  \end{equation}
  \begin{equation}
    \mathcal{D}_1 = \tilde{\phi}_{\bar{d}} + \theta\cdot\chi_{\bar{d}} + \frac{1}{2} \theta\cdot\theta F_{\bar{d}}
  \end{equation}
  \begin{equation}
    \mathcal{Q}_1 = \col{\tilde{\phi}_u}{\tilde{\phi}_d} + \theta\cdot\col{\chi_u}{\chi_d} + \frac{1}{2}\theta\cdot\theta \col{F_u}{F_d}
  \end{equation}
  \begin{equation}
    \mathcal{H}_u = \col{H^+_u}{H^0_u} + \theta\cdot\col{\tilde{\chi}^+_u}{\tilde{\chi}^0_u} + \frac{1}{2}\theta\cdot\theta\col{F^+_u}{F^0_u}
  \end{equation}
  \begin{equation}
    \mathcal{H}_d = \col{H^+_d}{H^0_d} + \theta\cdot\col{\tilde{\chi}^+_d}{\tilde{\chi}^0_d} + \frac{1}{2}\theta\cdot\theta\col{F^+_d}{F^0_d}
  \end{equation}
  where the numerical index refers to the generation number.

  It is interesting to note that all the particles are doublets and the antiparticles are singlets.

  We have to note that unlike the SM, we now have 2 distinct Higgs doublets, instead of just the $H$ and $H^\dagger$ from before. This is because having $H^\dagger$ would end up breaking supersymmetry as a whole so we had to find a way circumvent it. We did this by having $\mathcal{H}_u$ and $\mathcal{H}_d$ act like $H$ and $H^\dagger$ without the explicit need for the hermitian conjugation.

  Moving onto the gauge vector superfields, we have 3 fields to deal with:
  \begin{equation}
    \mathcal{B} = \frac{1}{2} \theta \sigma^\mu \bar{\theta} B_\mu + \left(\frac{1}{2\sqrt{2}} \theta\cdot\theta \bar{\theta}\cdot\bar{\lambda}_\gamma + h.c. \right) - \frac{1}{8}\theta\cdot\theta \bar{\theta}\cdot\bar{\theta} D_\gamma
  \end{equation}
  \begin{equation}
    \mathcal{G}^a = \frac{1}{2} \theta\sigma^\mu \bar{\theta} G^a_\mu + \left(\frac{1}{2\sqrt{2}} \theta\cdot\theta \bar{\theta}\cdot\bar{\lambda}^a_C + h.c.\right) - \frac{1}{8}\theta\cdot\theta\bar{\theta}\cdot\bar{\theta}D^a_C
  \end{equation}
  \begin{equation}
    W^i = \frac{1}{2} \theta\sigma^\mu \bar{\theta} W^i_\mu + \left(\frac{1}{2\sqrt{2}} \theta\cdot\theta\bar{\theta}\cdot\bar{\lambda}^i_L + h.c. \right) - \frac{1}{8} \theta\cdot\theta \bar{\theta}\cdot\bar{\theta} D^i_L
  \end{equation}

  \subsection{Building the Lagrangian}
  \label{ch:15:building the mssm:lagrangian}
  To form the Lagrangian, remember that we simply need to extract out either the $F$ or $D$ terms from the fields and we are done. It is a very simple task and all we really need to do then, is to find out which invariant superfields we can put into our Lagrangian.

  \subsubsection{Free Lagrangian}
  \label{ch:15:building the mssm:lagrangian:free}
    Let us work on the easiest part of the Lagrangian -- the free particles. For left chiral gauge superfields, we know that
    \begin{equation}
      \mathcal{L}_{Free} = \left.\Phi^\dagger \exp[2q\mathcal{V}] \Phi \right\vert_D
    \end{equation}

    For example, the quark doublet will be
    \begin{eqnarray}
      \mathcal{L}_{Free, Q_i} &=& \left.Q_i^\dagger \exp\left[\frac{1}{3} g^\prime \mathcal{B} + g_S \mathcal{G}^a \lambda^a + g \mathcal{W}^i \tau^i \right] Q_i \right\vert_D \nonext
      &=& D_\mu \tilde{\phi}^\dagger D^\mu \tilde{\phi} + i \chi^\dagger \bar{\sigma}^\mu D_\mu \chi + F^\dagger F - \left(\frac{\sqrt{2}}{6} g^\prime \tilde{\phi}^\dagger \lambda_\gamma \cdot \chi + h.c. \right) - \frac{1}{6} g^\prime \tilde{\phi}^\dagger D_\gamma \tilde{\phi} \nonext
      && - (\frac{1}{\sqrt{2}} g \tilde{\phi}^\dagger \chi\cdot\lambda^i_L \tau^i + h.c.) - \frac{1}{2} g \tilde{\phi}^\dagger \tau^i \tilde{\phi} D^i_L - (\frac{1}{\sqrt{2}} g_S \tilde{\phi}^\dagger \chi\cdot\lambda^a_C \Lambda^a + h.c.) \nonext
      && - \frac{1}{2} g \tilde{\phi}^\dagger \Lambda^a \tilde{\phi} D^a_C
    \end{eqnarray}
    where we have taken the liberty to express the doublets by as a single particle for obvious reasons. The same will go for the leptons and Higgs field, except that they would be much less complex than this since they do not have all 3 gauge charges.

    The free part of the gauge bosons is also very straightforward:
    \begin{equation}
      \mathcal{L}_{Free, GB} = \left. \frac{1}{4} \mathcal{F}_\gamma \cdot \mathcal{F}_\gamma \right\vert_F + \left. \frac{1}{2} Tr \left( \mathcal{F}_L \cdot \mathcal{F}_L \right) \right\vert_F + \left. \frac{1}{2} Tr \left( \mathcal{F}_C \cdot \mathcal{F}_C \right) \right\vert_F
    \end{equation}

    \subsubsection{Interactions}
    \label{ch:15:building the mssm:lagrangian:interactions}
    The hard part now is to find the terms that we may include as interactions. We need to ensure that the invariance conditions from before are all met. Moreover, let us also define a new dot product between doublets. $A \circ B \equiv A^T i \tau_2 B$ where both $A$ and $B$ are $SU(2)_L$ doublets. Doing so will ensure the $SU(2)$ invariance of the product and allows us to use it as possible interaction terms. Note that if we were to expand it out in full, we would see that $A \circ A = 0$.

    After going through all the possible permutations, we will be left with 7 non-vanishing invariants.
    \begin{itemize}
    \item $\mathcal{U}_1 \mathcal{Q}_1 \circ \mathcal{H}_u$
    \item $\mathcal{D}_1 \mathcal{Q}_1 \circ \mathcal{H}_d$
    \item $\mathcal{E}_1 \mathcal{L}_1 \circ \mathcal{H}_d$
    \item $\mathcal{H}_u \circ \mathcal{H}_d$
    \item $\mathcal{D}_1 \mathcal{Q}_1 \circ \mathcal{L}$
    \item $\mathcal{H}_u \circ \mathcal{L}_1$
    \item $f^{abc}_C \mathcal{U}_1^a \mathcal{D}_1^b \circ \mathcal{D}_1^c$
    \end{itemize}
    where the list here is only shown for the 1st generation of leptons and quarks. The actual list extends to all 3 generations of them.

  \subsection{The need for a new gauge}
  \label{ch:15:building the mssm:need for new gauge}
  The problem with the list is that the last 3 violate the conservation of lepton and quark numbers, something we know to hold empirically. To handle that, we can add in a new global gauge. This global gauge will only allow an odd number of superpartners to be in any invariant combination, which as we shall see, only holds for the first 4 combinations.

  To see why that is the case, let us look at how the superfields will expand when we take the $F$ terms.
  \begin{equation}
    \left.\mathcal{X}\mathcal{Y}\mathcal{Z}\right\vert_F = \phi_x \phi_y F_z + \phi_x \chi_y \cdot \chi_z + \cdots
  \end{equation}

  If we were to expand out all 7 terms to match the form above, we would see that the `good' interactions (i.e. the first 4 interactions) all have even numbers of superpartners, whereas the `bad' interactions have odd numbers of superpartners. If we introduce a $U(1)$ gauge symmetry to the Grassmann numbers, $\theta \rightarrow \exp[i\varphi] \theta$, we would see that having the superfields transform as $\Phi \rightarrow \exp[ik\varphi] \Phi$, our individual fields transform as
  \begin{equation}
    \phi \rightarrow \exp[ik\varphi] \phi \quad , \quad \chi \rightarrow \exp[i(k-1)\varphi] \chi \quad , \quad F \rightarrow \exp[i(k-2)\varphi] F
  \end{equation}
  To make things simpler, we may set $\varphi = \pi$ and we would see that we have a symmetry of parity!

  In taking the $F$ term, we would be integrating over $\int \td^2 \theta$, which brings with it a phase of $+1$. For the other terms to be invariant, it is necessary for them to transform with a phase of $+1$ as well. Doing so will naturally embed the condition that we discard the last 3 terms of our list.

  In total, our MSSM superpotential is
  \begin{equation}
    \mathcal{W} = y^{ij}_u \mathcal{U}_i \mathcal{Q}_j \circ \mathcal{H}_u - y^{ij}_d \mathcal{D}_i \mathcal{Q}_j \circ \mathcal{H}_d - y^{ij}_e \mathcal{E}_i \mathcal{L}_j \circ \mathcal{H}_d + \mu \mathcal{H}_u \circ \mathcal{H}_d
  \end{equation}
